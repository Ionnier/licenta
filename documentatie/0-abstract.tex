\begin{abstractpage}

\begin{abstract}{romanian}
Microserviciile sunt o arhitectură de sisteme distribuite ce a luat amploare în urma 
structurării organizațiilor către piață. 
Aceasta a fost facilitată de schimbări în modul în care lansăm produsele prin 
dezvoltarea platformelor cloud și a ce a dus la îmbunătățiri la timpul de 
livrare dar și a consistenței. Obiectivul este prezentarea acestor concepte și 
implementărea acestora prezentând modul de gândire ce influențează ciclul produsului. 
În acest scop, mă documentez legat de domeniu și încerc 
să îmi fac o idee de ansamblu asupra aspectelor ce trebuie luate în construirea unui sistem 
iar apoi o să aplic aceste informații. Aceasta lucrare ar trebui să servească ca un prim 
contact cu microserviciile dar și a practicilor DevOps.
\end{abstract}

\begin{abstract}{english}
Lorem ipsum dolor sit amet, consectetur adipiscing elit. Fusce vitae eros sit amet sem ornare varius. Duis eget felis eget risus posuere luctus. Integer odio metus, eleifend at nunc vitae, rutrum fermentum leo. Quisque rutrum vitae risus nec porta. Nunc eu orci euismod, ornare risus at, accumsan augue. Ut tincidunt pharetra convallis. Maecenas ut pretium ex. Morbi tellus dui, viverra quis augue at, tincidunt hendrerit orci. Lorem ipsum dolor sit amet, consectetur adipiscing elit. Aliquam quis sollicitudin nunc. Sed sollicitudin purus dapibus mi fringilla, nec tincidunt nunc eleifend. Nam ut molestie erat. Integer eros dolor, viverra quis massa at, auctor.
\end{abstract}

\end{abstractpage}