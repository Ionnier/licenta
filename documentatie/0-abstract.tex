\begin{abstractpage}

\begin{abstract}{romanian}

Microserviciile reprezintă o arhitectură de sisteme distribuite ce permite aplicațiilor
de a reacționa la diferite fluctuații precum numărul de utilizatori sau probleme de server ce cauzează
timpi morți, acestea devenind scalabile și reziliente. Kubernetes reprezintă o unealtă
ce poate fi folosită pentru orchestrarea microserviciilor.

Obiectivul acestei lucrări reprezintă explorarea noțiunilor teoretice ce formează
această arhitectură dar și despre Kubernetes, alături de moduri în care acestea pot fi folosite.

În plus, se vor urmării și alte elemente ce influențează subiectul, precum
serviciile oferite de platformele de infrastructură în cloud prin intermediul căreia
putem să accelerăm dezvoltarea dar și elemente de DevOps ce au ca scop eficientizarea
acesteia.

Ulterior, aceste concepte vor fi aplicate în implementarea unei aplicații pe baza
unor specificații și obiective.

\end{abstract}

\begin{abstract}{english}

Microservices are a distributed systems architecture that allows applications to
to react to different fluctuations such as the number of users or server problems causing
downtime, making them scalable and resilient. Kubernetes is a tool
that can be used to orchestrate microservices.

The objective of this paper is to explore the theoretical notions that form
this architecture but also about Kubernetes, along with ways in which they can be used.

In addition, other elements influencing the subject will be followed, such as
services offered by cloud infrastructure platforms through which
accelerate development, but also elements of DevOps that aim to make development more efficient.
it.

Subsequently, these concepts will be applied in the implementation of an application based on
specifications and objectives.

\end{abstract}

\end{abstractpage}