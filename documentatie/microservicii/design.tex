Partea de creare a schemei arhitecturale a unui sistem este una dintre
cele mai importante faze ale dezvoltării. Acesta este momentul în care putem descoperii
că microserviciile nu ar fi chiar o alegere potrivită pentru cerințele noastre. Din acest
motiv este important să nu ignorăm potențiale probleme ci să se formuleze toate problemele
pe care le putem aștepta în implementare

\subsection{Interconectare}
Deși microserviciile pot fi lansate și pot funcționa independent, acestea trebuie să funcționeze
în cadrul unui sistem complex. Astfel trebuie definite modul de comunicare și interfețele expuse.
În definirea interconectării întâlnim câteva concepte de bază prezentate anterior ce provin
din descompunerea în module.

Ascunderea de informații este procesul de separare al detaliilor de implementare pentru a evita
modificarea în alte locuri în cazul în care acestea se schimbă. Acest concept ofer varii beneficii
precum îmbunătățirea timpului de dezvoltare datorită posibilități de dezvoltare în paralel,
creșterea întelegerii sistemului mult mai ușor întrucât fiecare modul poate să fie înțeles
independent și flexibilitate oferit de faptul că putem schimba funcționalitate fără a fi necesar
ca alte părți ale sistemului să fie modificate. David Parnas spune că „The connections between modules are the assumptions which the modules make about
each other.” (Conexiunile create între module sunt ipotezele create de module față de celelalte) \cite{DBLP:conf/ifip/Parnas71}.
Astfel, cu cât alte microservicii folosesc un anumit microserviciu mai mult se crează mai multe
locuri în care devine mai dificil de modificat, astfel cu cât întâmpinăm problema aceasta mai puțin
cu atât putem lansa acel microserviciu fără a lua în considerare modificări care să pastreze compatibilitatea
cu elementele existente.

Un alt concept important este coeziunea, scopul microserviciilor este de a funcționa independent,
astfel vrem ca lucrurile care înteracționează să fie grupate iar atunci când se schimbă ceva 
să nu trebuiască să modificăm multe locuri. Elementele care nu interacționează ar trebui să fie separate
pentru a nu crea probleme de regresie.

În sisteme distribuite apare cuplarea, atunci când un microserviciu depinde mai mult sau mai puțin de
altul pentru a avea funcționalitate completă. Un sistem cu cuplare mică poate să fie lansat fără să trebuiască
modificări în locuri multiple.

Cele două concepte definesc lucruri asemănătoare, astfel ele se pot rezuma ușor ca „A structure is stable if cohesion is strong and coupling is low.” \cite{endres_rombach_2003} 
(O structură este stabilă dacă coeziunea este ridicată și cuplarea este scăzută). Coeziunea se referă la lucrurile
din interior iar cuplarea se referă la exterior iar acest lucru este greu de balansat.

Cuplarea poate apărea sub diferite forme, cuplare de domeniu în care un microserviciu interacționează
cu alt microserviciu pentru că are nevoie de funcționalitate ce depăsește atribuțiile lui. Astfel de 
cuplare, în general, nu poate fi evitată într-un sistem distribuit. Acesta este cel mai scăzut nivel
de cuplare însă trebuie să evităm cazul în care un microserviciu depinde un centru pentru alte microservicii
întrucât ar însemna ca acesta are prea multe responsabilități. Există conceptul de cuplare temporară, adică
atunci când un microserviciu așteaptă ca alt microserviciu să execute ceva, astfel cuplarea se întâmplă pentru că cele
două operațiuni se întâmplă în același timp. Cuplarea Pass-Through apare atunci când un microserviciu este nevoit
să trimită date unui alt microserviciu întrucât acesta trebuie să trimite datele 
unui microserviciu la un nivel mai mic, astfel un microserviciu are rolul doar de a transmite date. Acest lucru
este problematic întrucât dacă formatul acestor date ar trebui să se schimbe, ar trebui să
modificăm mai multe microservicii. Evitarea acestui tip de cuplare se face prin schimbarea modului
prin care comunicăm de exemplu microserviciul ce emite date ar putea să trimită direct datele către microserviciul ce are
nevoie de ele sau ca microserviciul interpediar să trimită datele in format pur, fără să îl intereseze o formă anume.
Cuplarea comună apare atunci când mai multe microservicii încearcă să acceseze o resursă 
comună, de exemplu o bază de date sau locația unui fișier. Problema în acest caz este că dacă modul în care
această resursă interacționează se schimbă, mai multe microservicii trebuie modificate. Sau dacă
unul din microservicii are nevoie de un comportament special, alte microservicii ar trebui să 
se adapteaze ceea ce crează probleme în dezvoltare. Soluții posibile de evitare pot fi crearea unui alt 
microserviciu ce funcționează ca o interfață pentru resursa comună ce va funcționa dupa principiile unui
microserviciu. Cuplarea de conținut apare atunci când un microserviciu modifică starea internă a altui 
microserviciu, de exemplu prin modificarea bazei de date a acesteia. Acest lucru este periculos întrucât pot
apărea modficări nedorite prin utilizare incorectă, astfel se încalcă procesul de ascundere de informații.

\subsection{Domain Driven Design}
Domain Driven Design este un concept introdus de Eric Evans în cartea sa \cite{ddd}. Acesta se bazează
pe crearea codului în jurul afacerilor pentru ca acesta să fie mai ușor de înțeles prin creșterea
interacțiunii cu analiștii de afaceri, acest lucru este benefic pentru microservicii întrucât acestea se 
pot orienta către mediul de afaceri pentru a se dezvolta alături de acesta fiind mult mai eficient.

Unul din concepte este limbajul universal ce cuprinde ca codul din interiorul unui microserviciu să 
folosească termeni folosiți de către utilizatori și de către analiști, acest lucru este benefic întrucât
face comunicarea mult mai eficientă, o persoană nou venită poate să înțeleagă afacerea și codul în același timp.
Dacă cele două sunt diferite, atunci pot exista probleme în comunicarea problemelor.

În DDD, un agregat este o unitate ce are o anumită stare ce se poate modifică în timp și are ca scop
modelarea unui concept din domeniul real. În general un agregat este cuprins în întregime de către
un microserviciu chiar dacă acesta poate conține mai multe microservicii. Un agregat poate fi schimbat
din exterior însă poate să își controleze singur starea pentru a preveni stările ilegale

Mărginirea contextului se referă la ascunderea mai multor activități ce țin de implementare
cu scopul de a defini anumite responsabilități sigure pe care le îndeplinește. Acest lucru se
poate îndeplini prin ascunderea modelului, atunci când vrem să expunem o entitate nu trebuie să
oferim toate informațiile despre acesta ci doar cele necesare. Distribuirea unui model, adică a
unei entități între mai multe microservicii se face prin traducerea acestei entități în domeniu propriu.

DDD în microservicii este un concept fundamental întrucât ambele pun la centru afacerea și
dezvoltarea în jurul ei. Astfel conceptele prezentate în DDD pot fi aplicate microserviciilor
mărginirea contextului are ca bază ascunderea de informații și duce la crearea de formate standard.

\subsection{Concepte de programare orientată obiect în microservicii}

Multe din problemele cu care ne confruntăm în microservicii pot fi tratate folosind concepte
ce au ca scop ușurarea modului în care creăm sisteme prin intermediul claselor în programarea 
orientată obiect. Deși diferite în natură, ambele au ca rol definirea unor interacțiuni între
elemente ce sunt făcute să funcționeze separat sau prin intermediul unuor interfețe.

Începând cu conceptele de bază, abstractizarea se referă la ascunderea implementărilor interne
și afișarea doar facilitățile oferite de clasă ce pot fi accesate din exterior. Acest lucru face 
referire la conceptul de ascundere al informațiilor descris anterior.

Encapsularea se referă la agregarea de date și atribute, adică funcționalități pentru a crea un tot
unitar, acest principiu este compatibil cu microserviciile și nevoia acestora de a funcționa
independent folosind datele proprii și permite o securizare mai bună a stării interne.

Moștenirea se referă la capacitatea unei clase de a prelua atributele și funcționalitățile unei clase
deja existente, deși mult mai greu de imaginat în microservicii, moștenirea în microservicii ne permite
să creăm microservicii ce extind pe cel existent fără a cauza probleme.

Polimorfismul este un concept ce face referire la capacitatea accesarii a unor multiple entități
prin intermediul unei singure interfețe, în contextul microserviciilor putem spune că un microserviciu
ar putea să înlocuiască alt microserviciu dacă are aceași interfață de comunicare și duce la
crearea a unor aceleași rezultate.

Pentru microservicii putem aplica și concepte mai complicate de OOP precum principiile SOLID.
Principiile SOLID sunt un set de reguli cu scopul de a oferi dezvoltatorilor o metodă de a
scrie cod, astfel încât acesta să respecte cerințele actuale și să permită dezvoltări/modificări
ulterioare cu ușurință fără a avea dificultăți cu metodele deja existente. SOLID este un acronim
iar fiecare literă face referire la un principiu. Principiul responsabilității unice (Single Responsible
Principle) se referă la necesitatea ca fiecare clasă și metodă să aibă o singură funcționalitate,
făcând asta reducem complexitatea dar și eventuale redundanțe create prin folosirea aceluiași
cod în mai multe locuri făcând ca fiecare porțiune din cod să aibă un singur motiv de a exista.
Principiul Deschis – Închis (Open Close Principle) conduce la faptul că entitățile scrise trebuie să
fie deschise pentru moștenire dar să nu poată fie modificate, astfel asemănător cu principiul
anterior, codul deja scris nu va fi afectat întrucât noile modificări se aduc asupra unei porțiuni
noi. Principiul substituției al lui Liskov (Liskov Substitution Principle) se referă la faptul că clasele
derivate trebuie să acopere funcționalitățile clasei pe care o moștenesc astfel încât acestea să
poată să le înlocuiască fără modificări adiționale. Principiul separării interfețelor (Interface
Segregation Principle), clasele ce implementează interfețe nu ar trebui să implementeze metode
de care nu au nevoie, astfel dacă este nevoie să adăugăm o metodă nouă unei interfețe aceasta
trebuie să fie implementată doar de către obiectele care au nevoie, dacă acest fapt nu este
satisfăcut trebuie ca interfețele să fie separate ca să nu forțăm alte clase să implementeze
metode noi. Principiul inversării dependințelor (Dependency Inversion Principle) se referă că
modulele de nivel înalt nu ar trebui să depindă de cele de nivel scăzut, ele trebuie să fie
abstractizate astfel acestea pot fi modificate în orice moment dacă vrem să aducem un alt modul
cu funcționalități similare dar cu implementări diferite.\cite{temasolid}

\subsection{Migrarea}

În planificarea urbană există conceptul de dezvoltare pe spațiu verde. Aceasta se referă 
la dezvoltarea pe un loc în care la momentul începerii lucrărilor este spațiu varde ce nu necesită
costuri adiționale pentru curățare. Atunci când vrem să construim un sistem bazat pe microservicii
pornind de la zero, apare acest fenomen. Însă de multe ori există un sistem existent ce ajunge
la anumite limitări ce fac dorința de trecere către microservicii. Acest lucru face referire la 
conceptul de dezvoltare pe spatiu maro, adică pe un spațiu post industrial în care există costuri de 
demolare sau restaurare și ce necesită o atenție mult mai ridicată datorită posibilității alterării
compoziției pământului ce ar putea să fie toxic.

Migrarea unei aplicații monolitice nu ar trebui să fie făcută cu scopul a obținerii unei aplicații
formate din microservicii ci pentru că există o funcționalitate ce este împiedicată de a
funcționa la capacitățile necesare pentru funcționare optimă, astfel migrarea unui sistem 
monolitic ar trebui să fie făcută incremental și posibil nu complet dacă nu este nevoie.

Începutul migrării ar trebui să fie făcut cu o analiză a sistemului și identificarea dacă 
avem posibilitatea de a întreține microserviciile, acest lucru înseamnă să avem prezentă infrastructură și
oameni care să știe să configureze uneltele necesare precum cele de CI/CD pentru a livra cu aceași consistență.
Ulterior putem analiza sistemul pentru a observa locurile unde se formează coeziune fapt
ce duce la posibilitatea creării unui microserviciu cu cod grupat. Pentru acest lucru se pot folosi 
unelte precum CodeScene ce pot identifica porțiunile de cod ce se schimbă împreună. Acest lucru 
ajută în crearea unui context mărginit și să identificăm modul în care acesta ar funcționa cu exteriorul.

O dată ce identificăm lucrurile ce vrem să le separăm putem separa funcționalități și elementele 
conținute, UI-ul, backend-ul și baza de date. Am fi tentați să ignorăm o parte din aceste componente
însă microserviciile funcționează cel mai bine atunci când toate acestea sunt separate prin simplul motiv
ca chiar daca UI-ul nu este 1:1 cu backend-ul, acestea tot vor fi modificate în același timp. În acest scop,
există două modalități în care putem face migrarea, pornind de la cod, însemnâand că separăm funcționalitatea
însă în continuare folosim datele din baza de date monolitică. Acest tip de abordare este destul de simplu,
întrucât obținem rezultate bune pe termen scurt, însă există limitări întrucât este posibil să nu
putem separa datele din baza de date, fapt ce înseamnă că munca depusă pentru pentru separare ar fi
inutilă. Separarea datelor mai întâi se referă la crearea unei baze date separate pentru datele 
pe care vrem să le folosim, acest tip de abordare aduce mai multe probleme la început întrucât 
ne face să ne gândim la probleme precum pierderea integrității datelor.

Atunci când facem modificări de tipul acesta trebuie să avem grijă cum orchestrăm schimbările.
O modalitate ce provine de la un arbust tropical „Strangler Fig” se referă la crearea unui strat
de intercepție ce verifică dacă funcționalitatea apelată a fost implementată într-un microserviciu,
și daca este atunci va ruta traficul către microserviciu. Acest tip de implementare permite să nu
modificăm aplicația monolitică și să putem să ne întoarcem la ea în cazul în care avem probleme.
De asemenea, dacă datele sunt comune putem folosi load balancing între cele două la nevoie.
În rularea paralelă, microserviciul și aplicația monolitică funcționează în paralel și ne permite să
le comparăm ca să ne asigurăm că avem funcționalitatea dorită. Metoda comutatorului ne permite
să avem cele două sisteme aplicația monolitică și microserviciul să funcționeze în același timp și să 
comutăm între cele două foarte ușor printr-un proxy.

În momentul în care începem separarea datelor, apar probleme ce nu pot fi prezise cauzate de lipsa
unei baze de date monolitice. De exemplu, performanța unei baze de date este extrem de bună când este vorba
de crearea de cereri ce conțin operații de alăturare de tabele, astfel noi trebuie să mutăm aceste operații
extrem de optimizate în interiorul microserviciilor. O dată cu lipsa unei baze de date comună pentru toate
tabelele apar problema lipselor de constrângeri, nu ne mai putem baza pe baza de date să 
forțeze aceste constrângeri și trebuie să verificăm și să tratăm erorile în cazul în care apar erori.
Tranzacțiile nu mai sunt garantate. O dată cu separarea datelor se pierde ACID-itatatea, o altă problemă 
a microserviciilor atunci când avem tranzacții distribuite între mai multe microservicii. 
În funcție de necesități ar fi necesar să rulăm cereri direct în baza de date sau să oferim acces
către persoane din exterior către aceasta, însă prin conceptul
de ascundere de informație acest lucru nu ar fi posibil. Ca să rezolvăm această problemă putem crea o copie
a bazei de date, read only, pe care microserviciul o va popula o data ce populeaza baza de date inițială.

Astfel, migrarea unui sistem monolitic este extrem de dificilă și vor apărea probleme, din acest
motiv este recomandat să folosim o abordare incrementală în care mutăm funcționalități una câte una,
acoperindu-ne obiectivul pe care ni-l dorim prin introducerea arhitecturii formate din microservicii.
