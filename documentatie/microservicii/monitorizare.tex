Necesitatea monitorizării nu apare atunci când dezvoltăm microserviciile ci atunci când le lansăm.
Atunci când apare o problemă nereproductibilă, singurul lucru ce ne poate descrie modul în care am ajuns în
această situație sunt urmele pe care le creăm intenționat, adică fișierele de jurnalizare
sau tranzacțiile care s-au desfășurat. Într-o aplicație monilitică acesta are un număr mic 
de locuri în care poate să cedeze, astfel investigațiile pornesc mereu din același loc. Monitorizarea
acestuia este mai simpla întrucât numărul de resurse pe care acesta le apelează este mai mic.
În cazul unei arhitecturi distribuite atunci când apare o problemă, acleași investigații
pe care le-am face pe o platformă monolitică o să le facem asupra mai multor componente.
Din acest motiv trebuie să găsim metode de monitorizare eficiente ce sunt ușor de accesat
și folosit pentru cei implicați

Implementarea unui astfel de sistem în interiorul unui microserviciu depinde de modul în care
acesta comunică. În cazurile simple în care avem o singură instanță al unui microserviciu
atunci monitorizarea acestuia este simplă întrucât trebuie să colectăm date dintr-un singur loc.
Atunci când acest microserviciu este scalat și există mai multe instanțe al acestuia care
pot să apară sau să dispară în funcție de load, atunci trebuie implementat un sistem
care colectează datele de la acestea și eventual le asociază datele asupra cererilor. Atunci când
aceste microservicii comunică și cu exteriorul devine și mai complicat întrucât nu vom știi
exact modul în care sistemul va acționa, în acest caz agregarea informațiilor nu este suficientă
și avem nevoie de metode de prelucrare eficientă al acestora.

Observabilitatea este capacitatea de a întelege starea unui sistem în funcție de datele pe care
acesta le comunică, asemănător cu bordul unei mașini ce ar lumina anumite beculețe în funcție 
de problemele pe care mașina ar putea să le întâmpine în interior. Aceste date furnizate pot
ajută în prevenirea unui eventuale probleme, fie prin scalarea automată a sistemului în cazul în care
datele furnizate de acesta indică că este suprasolicitat fie prin identiciarea problemelor 
înainte să apară prin crearea de algoritmi bazați pe inteligența artificială, iar toate acestea
ar trebui să se vadă dintr-un loc central pentru a putea examina starea sistemului în același 
mod pentru toate componentele acestuia.

În acest scop, atunci când creăm un sistem bazat pe microservicii ar trebui să pornim de 
la construirea unei baze pentru observabilitate, astfel putem avea în vedere anumite 
caracteristici care trebuie să apară. Una dintre cele mai importante și simple de implementat
acțiuni este agregarea log-urilor. Dacă putem crea asta atunci putem să construim un istoric
al acțiuniilor ce s-au declanșat în cadrul sistemului. Deși există unelte dedicate, toate
ar lucra în același stil, microserviciile crează fișiere de jurnalizare local pe care alt componentă
le va colecta și le va trimite către componenta de agregare. Este important ca acesta să nu există procesări mari 
ale acestor linii întrucât cantiatea de informație care apare poate să creeze creșteri mari în
consumul de resurse. Fiecare intrare în aceste fișiere ar trebui să fie standardizată cu informațiile 
de care avem nevoie, de exemplu data, microserviciul, mașina, nivelul de importanță al log-ului și 
atle date de care am avea nevoie în depanare. Toate aceste date sunt folosite de agregator pentru 
a crea centralizări informațiilor sau pentru a cunoaște momentele în care ceva grav se întâmplă
și să fie nevoie să alertăm personalul. Întrucât microserviciile funcționează pentru rezolvarea
unei singuri cereri, este important să cunoaștem cererea pentru care se lucrează,
în general se asociază un ID de corelare pe care fiecare microserviciu îl va comunica 
pentru ca tranzacțiile din cadrul cererii să fie ușor de interogat atunci când acestea vor fi
agregate. Deși fiecare înregistrare are o dată, aceasta nu poate să fie sigură iar încrederea în 
acestea poate fi o capcană. În programare, cel puțin pentru mine, lucrul cu timpul este cel mai dificil,
în interfață deoarce există fusuri orare ce trebuie tratate (desigur e ușor să salvăm datele 
în UTC și doar să le afișăm în mod diferit, însă se adaugă și schimbările create de economiserea
de economiserea de lumină), iar în cadrul fișierelor de log ale microserviciilor 
timpul nu poate fi precis întrucât fiecare are timpul său intern, deși poate să fie sincronizat
cu un server de sincronizare Network Timing Protocol (NTP) nu poate să fie exact. Putem folosi 
unelte precum Fluentd ce trimite date în Elasticsearch cu interfață în Kibana însă acest spațiu
este populat cu soluții. Atunci când avem un sistem de acest tip, pot apărea probleme, pe lângă ce a
fost menționat, apar costuri cu privire la stocarea datelor dar și necesitatea unei puteri computaționale
în creștere pentru a menține index-urile coloanelore din bazele de date.

Un alt aspect ce comunică starea sistemului sunt metricile și telemetria. Datele colectate de 
acestea pot include procentul de resurse consumate, arhitectura sistemului, modul cum răspund la cereri 
sau chiar performanța acestora. Aceste metrici sunt în genera simple întrucât urmăresc 
stocarea unei singure informații sau acțiuni, însă acompaniate de acestea apar informații 
suplimentare iar acestea sunt costisitor de depozitat întrucât va trebui să fie interogate.
În acest scop putem folosi unelte precum Prometheus sau Graphite.

Colectarea datelor despre un sistem se face într-un singur scop, aprecierea dacă totul
funcționează în parametrii nominali. În cadrul unei aplicații monolitice, această apreciere 
era simplă întrucât lucrurile pot să meargă sau să nu meargă. Într-o arhitectură paralelă
ce se bazează pe independență această apreciere devine mai dificilă. Dacă un microserviciu rulează,
este un pas, însă nu este suficient. Acesta trebuie să fie accesibil pentru toți consumatorii 
acestuia. Însă dacă funcționarea acestui microserviciu depinde de alt microserviciu care 
momentan are probleme, putem să spunem că acest microserviciu este într-o stare bună?
În aceste cazuri se creează anumiți termeni, primul concept fiind Termeni la nivel de serviciu,
„Service level agreement” ce se crează între cei care construiesc un sistem și cei care folosesc
acest serviciu, astfel aceștia din urmă pot să își creeze un anumit grad de încredere sau
nivelul la care se așteaptă. De exemplu, providerii de infrastructură în cloud pot 
crea astfel de termeni pentru a garanta funcționalitatea serviciilor oferite de ei, pe
perioadele în care acestea nu pot fi accesibile din cauza lor, probabil nu vor exista credite 
consumate însă probleme ale sistemului nostru tot pot apărea ce pot afecta clienții proprii.
Obiectivele la nivel de serviciu („Service-Level Ojective) sunt create pentru a crea o imagine sau un posibil progres
în atingerea SLA-ului. Accestea pot include lucruri precum timp stabil sau latență. Pentru 
a aprecia atingerea SLO-ului putem crea Indiccatori la nivel de serviciu („Service-level Indicators”)
ce depind de sistemul oferit. În construirea unui sistem putem să alocăm spațiu pentru erori, 
acesta este scopul SLA-urilor, întrucât nu putem să fim perfecți ar trebui să păstrăm
loc și pentru apariția unor erori. Acestea ajută în crearea deciziilor care includ riscuri 
precum înceracare de tehnologii noi.

Apariția erorilor este inevitabilă, din acest motiv trebuie să întelegem modul în care
acestea pot să fie raportate. Există erori al căror apariție poate fi ignorată pe moment, în 
general la aceste tipuri de erori ne putem aștepta și au o severitate redusă. În interiorul unui sistem
trebuie să știm tipurile de erori care impactează funcționalitatea sistemului și trebuie remediate
în cel mai scurt timp, în general pentru cazurile acestea un tehnician ar fi alertat 
prin diferite forme pentru a remedia problema. Întrucât astfel de situații crează un grad 
mare de disconfort, trebuie să alegem erorile care generează acest tip alerte cu grijă,
întrucât dacă sunt prea multe e posibil ca acestea să fie ignorate. Studii create
pentru a studia motivele pentru care unele accidente apar pot să concluzioneze
că din cauza numărului mare de alerte raportate care nu au gradul corespunzător pot genera
pentru cei care le urmăresc o oboseală, din acest motiv aceștia pot să nu urmeze procedurile
corespunzătoare. Steven Shorrock menționează în articolul „Alarm Design: From Nuclear
Power to WebOps” (Humanistic Systems (blog), October 16, 2015) că motivul alertelor
este de a direcționa atenția utilizatorlui către aspecte ale operației ce au nevoie
de atenție într-un timp scurt. („The purpose of [alerts] is to direct the user's attention towards 
significant aspects of the operation or equipment that require timely attention.”). Din acest motiv
ne putem inspira din reguli vizează comportamente ce necesită utilaje complexe.
Asociația utilizatorilor de echipament și materiale (Engineering Equipment and Materials Users Association - EEMUA)
a creat o descriere pentru crearea alertelor într-un mod corect. Acestea trebuie să fie relevante,
să ilustreze o caracteristică importantă a sistemului, unice, pentru a evita repetiția, să ajungă 
într-un timp util în care acțiuniiile utilizatorilor mai pot crea un impact, sa conțină
suficiente informații pentru ca acesta să le poată prioritiza, informația ce provine din alerte
să fie clară și ușor de înteles, să conțină informațiile necesare pentru a începe o acțiune cu scopul
de fixare, dacă este posibil să ofere sfaturi pentru rezolvare, și să fie concentrate pe 
problema pe care o raportează.

Monitorizarea unui sistem este dificilă, însă aceasta poate să fie introdusă treptat însă 
putem începe cu agregare fișierelor de jurnalizare dar și notarea parametrilor în care rulează
mașinile de lucru și raportarea acestuia într-un loc central.
