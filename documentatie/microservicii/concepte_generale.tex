Microserviciile sunt o arhitectură de sisteme distribuite ce a devenit populară în ultimele decenii datorită
necesității scalării. Ceea ce o separă de o arhitectură orientate pe servicii este faptul că microserviciile
sunt mult mai specifice, acoperind o singură parte din afacere și faptul că pot fi lansate în execuție independent,
astfel microserviciile ar trebui să depindă cât mai puțin de altele. Comunicarea se face direct către interfețele
expuse de fiecare, iar informațiile reținute de fiecare (de exemplu baza de date de date sau diferite fișiere)
pot să fie modificate doar prin metodele de comunicare oferite ci nu direct.

Un concept important în microservicii este ascunderea de informații, astfel din exterior fiecare ar trebui să 
fie tratat ca o cutie neagră ce face anumite lucruri și expune anumite date, nu contează detaliile de implementare
și nici tehnologiile folosite în implementare, ci doar capabilitățile fiecăruia. Acest lucru permite să modificăm
microserviciul în funcție de cerințe, chiar și refactorizarea metodelor inițiale cât timp se respectă 
capabilitățile pe care microserviciul ar trebui să le îndeplinească.

Cel mai important aspect al microserviciilor este capacitatea de a lucra independent de celelalte. Acestea
trebuie să funcționeze independent de celelalte iar dacă aducem schimbări într-unul atunci nu suntem obligați să
facem schimbări în alt microserviciu, în caz contrat ar apărea dificultăți în lansare. Obținerea independenței poate
să aducă contribuții majore în timpul de deployment însă implementarea este mult mai complicată datorită necesității
comunicării cu alte servicii.

Un alt aspect de bază este acoperirea unui singur segment din afacere prin intermediul unui microserviciu. Alături 
de caracteristica precedentă, dacă afacerea și-ar dezvolta necesitățile ar fi mult mai dificil să modificăm
mai multe microservicii și ulterior să coordonăm lansarea lor.

Independența microserviciilor este importantă, în acest scop, ele ar trebui să fie singurele care ar putea să își modifice starea.
Întrucât starea se referă la elementele componente, de exemplu o bază de dată, acestea nu ar trebui să fie împărtășite
și nici accesate de alte microservicii. De asemenea ar trebui evitate modificarea interfețelor declarate pentru
a evita necesitatea modificării altor microservicii.

Dimensiunea microserviciile ar trebui să fie redusă prin oferirea unui număr restrâns de funcționalități declarate pentru a
a fi ușor de înțeles și întreținut. Însă important este cât de multe microservicii
pot fi administrate întrucât un număr mai mare de microservicii, deși mici ca funcționalități, aduc dificultăți
în comunicare, scalare și depanare.

Microserviciile sunt agnostice din punct de vedere al tehnologiilor care ar putea fi folosite. Din acest motiv, acestea
oferă flexibilitate însă prețul plătit este crearea a mai multor puncte vulnerabile.

Orice prezentare al acestui tip de arhitectură nu ar fi completă fără prezentarea conceptului de „monolith” (monolit), întrucât
este considerată o arhitectură veche reprezentând modul de funcționare din trecut. În cel mai simplist mod de funcționare, 
sistemul monolitic este reprezentat de un singur proces „gigantic” ce acoperă toată funcționalitatea sistemului, în general conectat
la o singură bază de date ce și ea este reprezentată de un singur proces. Putem intui problemele cu această organizare,
dacă conexiunea către procesul monolitic cade sau baza de date are probleme, întreg sistemul este afectat. Microserviciile incearcă 
să acopere aceste probleme prin independență, astfel dacă un microserviciu este căzut, celelalte nu sunt afectate.
Procesul poate fi însă separat pe module dezvoltate separat însă care la final se cuplează formând un singur proces.
Există conceptul de sistem monolitic distribuit în care acesta este la fel separat în module independente însă care nu ar funcționa
dacă nu sunt toate active, acest tip de sistem oferă toate dezavantajele sistemelor distribuite dar și a monolitului ceea 
ce îl face destul de rar întâlnit.
                  
Chiar dacă o arhitectură monolitică prezintă dezavantaje evidente ce sunt ușor de văzut, acestea nu sunt la fel de mari
pentru companiile mici. Numărul redus de aplicații ce trebuie lansate și monitorizate,
posibilitatea de reutilizare a codului mult mai ușoară, obținerea infrastructurii pentru
proces și uneori chiar și scalarea, deși este limitată poate să facă o arhitectură monolitică
mult mai atractivă. Nu ar trebui să asociem arhitectura monolitică ca un lucru antic, ci să 
o considerăm ca o opțiune. Însă pentru o organizație mică, acesta ar trebui să fie doar un
punct de plecare, eventual după ce serviciul oferit devine mai popular s-ar putea să ajungem
să cunoaștem limitările arhitecturii și să ne orientăm tot către microservicii, însă dacă 
nu avem succes, eliminăm multe complexități apărute în urma introducerii microserviciilor.

Principalele avantaje ale microserviciilor provin de la baza acestora, o arhitectură distribuită,
însă cuplat cu conceptul de ascundere a informației și domain-driven design (design orientat domeniu)
aduc multe alte avantaje asupra altor arhitecturi distribuite. În cadrul unui sistem format din microservicii
putem folosi orice tip de tehnologie pentru program și orice tip de bază de date, întrucât
ascundem implementarea de exterior. Astfel putem alege tehnologii ce aduc avantaje în 
dezvoltarea microserviciului. Acest avantaj duce la capabilitatea de a folosi tehnologii noi
fară a afecta tot sistemul, însă limitează capacitatea de a împărtășii cod (de exemplu prin
librării interne, întrucât acestea ar trebui să fie rescrice pentru fiecare limbaj). 
Un sistem distribuit rezistă mult mai ușor la căderi, întrucât acestea pot fi tratate ca să
se prevină un lanț, însă cu un număr mare de puncte slabe devine greu de aproximat cât de
mult este afectat un sistem. Scalarea unui sistem monolitic este ușoară, doar replicăm procesul,
însă la microservicii scalarea poate fi mult mai concentrată pe punctele care chiar au nevoie să fie
scalate. Lansarea unei schimbări într-un microserviciu este mult mai ușoară întrucât nu necesită
crearea de timp mort în aplicație, întrucât nu tot sistemul este înlocuit ci doar o mică parte din acesta,
ce poate fi chiar și mai controlată prin diferite aplicații de orchestrare.

Adaptarea microserviciilor vine cu un cost, pe lângă dificultățile de implementare a design-ului apar
și alte probleme, de exemplu dezvoltarea locală. Atunci când dezvoltăm un microserviciu acesta, posibil,
este nevoit să comunice cu alte microservicii care la rândul lor comunică cu alte microservicii și putem continua,
toate acestea microservicii trebuie să funcționeze în același timp pe laptop-ul dezvoltatorului,
întrucât nu putem folosi microserviciile din producție, însă nu e posibil să rulăm foarte multe
microservicii pe un singur computer depinzând de configurație. Adoptarea microserviciilor,
în general nu poate avea succes decât dacă este combinată cu elemente de DevOps, ceea ce înseamnă
ca dezvoltatorii și operatorii să învețe tehnologii noi, ceea ce este un împediment în momentul
în care vrem să lansăm un produs rapid în piață și nu avem experiența necesară. Microserviciile necesită
mult mai multă infrastructură pentru a fi rulată, mai multă tehnologie ce trebuie să fie folosită
pentru ca dezvoltarea și livrarea să decurgă eficient, însă pe termen lung acesta poate să
fie prevăzută dacă arhitectura este folosită cum trebuie prin livrare mai rapidă și mai eficientă
deci profitul ar putea să crească. Monitorizarea sistemelui devine mai complicată, într-un proces monolitic
toate fișierele de jurnalizare (logging) sunt în același loc. Într-un sistem distribuit, acestea 
trebuie să fie colectate, și ulterior ansamblate pentru a avea sens, acest lucru devine mai dificil când
intră în discuție fenomenul de replicare al unei instanțe. Securizarea unui sistem distribuit este
mai grea, traficul de date se face mult mai mult prin rețele sau prin diferite căi alternative ceea 
ce necesită un grad ridicat de atenție. Testarea unui sistem format din microservicii este mult mai
dificilă, mai ales când trebuie testate mai multe microservicii în același timp, acesta duce la creșterea
timpului necesar pentru a primi răspuns. Sistemele distribuite cresc timpul de răspuns întrucât acestea nu
se mai face în cadrul unui singur proces, deci datele trebuie codificate și decodificate atunci când trec
dintr-un mediu în altul. În cadrul unui sistem distribuit o altă problemă care apare este consistența 
datelor, dată de faptul că în timpul unei cereri, datele dintr-un microserviciu se pot schimba, astfel
tranzacțiile devin dificile mai ales când înainte ne bazam pe atomicitatea și consistența oferită de 
baza de date.

Astfel, microserviciile funcționează cel mai bine pentru organizațiile mari ce vor ca dezvoltatorii
lor să livreze facilități noi foarte repede care să nu fie întârziate de compatibilitatea cu 
caracteristici deja existente. Prin natura variată a acestei arhitecturi se pot folosi tehnologii noi
dar și folosirea platformelor cloud, astfel se oferă multă flexibilitate cu prețul creșterii 
complexității. Pentru companiile mici, al căror rată de succes este greu de calculat, overhead-ul 
produs de adaptarea unei arhitecturi în acest stil nu ar aduce suficient de multe avantaje și ar
trebui să opteze pentru o arhitectură tradițională ce este stabilă și oferă colaborare mult mai ușoară
în cadrul aceluiași proiect la scale mici. 
