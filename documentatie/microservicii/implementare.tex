Trecerea de la o aplicație monolitică la una bazată pe microservicii uneori nu este
ușoară. Unul din principalele motive este o posibilitate de a regândii modul cum accesăm
anumite elemente. Atunci când totul se află în cadrul unui singur proces, acessarea datelor
devine mai grea întrucât nu avem acces la ele și uneori nu este rentabil să căutăm după ele.
Atunci când apelăm o funcție local, timpul necesar pentru executarea acesteia este neglijabil,
atunci când această funcție trebuie să apeleze un alt microserviciu, lucrurile se complică și
trebuie decis dacă chiar avem nevoie. Un apel făcut printr-o rețea mereu va avea o întârziere, de asemenea
cresc locurile în care acesta poate eșua. În dezvoltarea aplicațiilor mobile întrucât 
ne așteptam ca aplicația să reacționeze la atingeri, nu putem să facem anumite operații pe
thread-ul principal, acestea pot fi apelarea lucrurilor de pe Internet sau accesarea unei baze
de date (chiar și dacă este de tipul SQLite3 fiind locală), aceasta pune în perspectivă 
impactul pe care acest tip de apeluri le pot avea.

Atunci când modificăm modul în care comunică un microserviciu trebuie să avem grijă la modul
cum ar afecta alte microservicii, sau atunci când acest lucru este necesar trebuie să orchestrăm
aplicarea acestor schimbări. Un alt concept ce într-o arhitectură monolitică se tratează ușor
este tratarea erorilor, întrucât toate se întâmplă în același mediu de lucru putem avea o viziune
asupra ce poate să cadă și în ce mod. Într-un sistem distribuit, pot apărea mai multe tipuri
de erori care se datorează rețelor, de exemplu crash-uri de microservicii ce necesită resetări,
erori de omisie de mesaje, când încercăm să apelăm un microserviciu însă acesta nu trimite un răspund,
în același timp acest răspund poate să fie mult prea rapid și să nu putem să îl recepționăm
sau prea târziu și să considerăm că nu am primit, dar și erori arbitrare precum unele în care 
ne așteptăm la un răspund și nu primim ce avem nevoie.

Toate aceste aspecte trebuie tratate atunci când lucrăm cu microservicii, însă fiind vorba
de o arhitectură paralelă ce oferă independență avem flexibilitatea de a alege ce folosim și ce nu.
De aceea este important să alegem tehnologii și stiluri de comunicare care oferă 
performanțele dorite încercând să minizăm timpul de răspuns și tipurile de avarii ce pot apărea.

\subsection{Tipuri de comunicare}

La fel ca în comunicarea între procese sau în cadrul unui proces, există comunicare sincronă
ce introduce blocare la un anumit nivel și comunicare asincronă ce permite procesele să 
trateze cererile în momentul în care pot, aducând îmbunătățiri ale perfomanței dar crescând
dificultatea întelegeri și a modurilor în care aplicația poate cădea.

Cel mai simplu de înteles mod de comunicare este comunicarea sincronă bazată pe cerere și răspuns.
În cadrul acestuia un microserviciu face o cerere(indiferent de formă) către un alt microserviciu și așteaptă
un răspuns. Acest scenariu este extrem de familiar, semănând cu cererile către o bază de date sau către
un API extern, ceea ce le face ușor de implementat însă nu este o tactică eficientă în cadrul unui sistem distribuit
întrucât aduc dezavantaje precum apariția acestui blocaj, creșterea latenței dar și apariția
acestui cuplaj temporar întrucât se poate întâmpla ca unul dintre microservicii să nu fie același
ca cel care a răspuns întrucât a fost necesară o resetare. Acest tip de comunicare poate fi folosită
pentru microservicii simple dar trebuie evitat crearea unui șir întreg de cereri întrucât crește
timpul total de răspuns dar și apariția unui eșec în cascadă.

Însă, o arhitectură paralelă are posibilitatea creării a unui tip de comunicare asincronă ce
nu blochează funcționarea microserviciilor, adică ele nu așteaptă un răspuns direct. Avantajul
unui tip astfel tip de comunicare este posibilitatea tratării operațiunilor ce ar dura mai mult 
întrucât se elimină conceptul de cuplare temporară. Dezavantajul este creșterea nivelului de complexitate
și creșterea dificultății de monitorizare și de a manevra erorile, însă ar trebui să fie 
utilizată atunci când avem procese de lungă durată sau pentru a comunica cu mai multe microservicii 
în același timp.

Cel mai simplu mod de a implementa comunicarea asincronă este crearea de resurse partajate.
Un microserviciu adaugă date, fie într-o locație de stocare, adăugăm un fișier sau la o bază de date
adăugăm alte date, iar celelalte microservicii ce vor să acționeze în urma primirii acestei informații
doar verifică apariția de date noi, iar atunci când apare o prelucrează. Dezavantaje includ
necesitatea tratării cazurilor în care microserviciile ce scanează sunt căzute, iar în momentul
în care vor reveni sunt nevoie să trateze date noi dar și faptul că se realizează o formă
de cuplare asupra domeniului partajat. Această formă de comunicare poate fi folosită atunci 
când suntem restricționati de tehnologiile folosite, întrucât orice tip de sistem poate să
citească date dintr-un fișier dar și atunci când vrem să trimitem un volum de date ridicat.

Un alt tip de comunicare este cel sub formă cerere-răspuns, acesta poate fi sincron sau asincron.
În mod blocant, microserviciul ce face cererea va astepta un răspuns iar în funcție de acesta 
își va continua activitatea. În mod asincron, se va proceda în mod asemănător însă de data aceasta nu se
va mai aștepta un răspuns, sau cel puțin nu unul cu informația cerută. Acest lucru este mai dificil,
întrucât pentru a răspunde cererii, microserviciul trebuie să trimită unui microserviciu
specific răspunsul creat și există posibilitatea ca microserviciul ce inițiază cererea să
nu mai existe astfel am putea încerca ca orice microserviciu să poată să trateze orice cerere.
Putem folosi acest mod de implementare atunci când avem nevoie să avem o confirmare la 
cererile făcute și de asemenea este destul de ușor de implementat, cel puțin varianta blocantă.

Una dintre comunicările reprezentative pentru microservicii este cea făcută prin evenimente,
este asincronă în natură și se bazează că atunci când în cadrul unui microserviciu 
se întâmplă un lucru pe care consideră că ar trebui să îl comunice în exterior, acesta 
emite un eveniment cu diferite informații astfel celelalte microservicii pot asculta
iar în momentul primirii acestea vor reacționa independent. Este un opus al modelului anterior
întrucât în acest tip de comunicare, în general, nu avem cunoștințe dacă evenimentul a ajuns
către celelalte părți sau dacă cineva ascultă pentru acest tip de evenimente
ci eventual dacă a fost procesat de către Event Bus. Comunicarea prin evenimente oglindește
modul de organizare al unor echipe autonome din cadrul companiilor, acestea anunță atunci 
când produc un rezultat nou și nu primesc feedback direct dacă s-a acționat după mesajul lor
în mod implicit, întrucât comunicarea poate fi un mail general la o listă de abonați, aceștia
ar primi doar confirmarea că serverul ce procesează mail-urile a validat cererea. Implementarea
constă în crearea unei metode de a trimite evenimente și de a recepționa. Un prim mod
este folosirea de programe specializate precum Apache Kaftka sau RabbitMQ ce livrează 
platforme dar și implementări pentru ambele probleme ce vin cu avantajul că toate problemele
care le-am avea precum garantarea livrării și permit scalare însă sunt un sistem în plus de 
administrat. Un alt mod de implementare include folosirea specificatiilor declarate in
Atom pentru a crea un flux web de resurse ce poate fi folosit pentru a trimite și consuma
evenimente însă nu oferă latență mică. În general, ar fi bine să nu implementăm comunicarea
prin evenimente de la zero întrucât este dificil de administrat și sunt multe probleme greu de
tratat, de aceea putem folosii produse existente. Avantajul unui astfel tip de comunicare
este posibilitatea ca un eveniment sa fie tratat de mai multe microservicii in acelasi timp,
astfel evenimentul trebuie să conțină informații suplimentare pentru ambele microservicii,
deci apare riscul de a distibui mai multe informații ceea ce induc diferite probleme de securitate,
astfel putem folosii acest tip de arhitectură acolo unde putem profita de acest avantaj.

Deși comunicarea asincronă aduce îmbunătățiri în decuplarea sistemelor și în timpul de răspuns
în cazul în care vrem să comunicăm cu mai multe evenimente acesta aduce complexități ce devin
greu de depanat, astfel avem nevoie de monitorizare și moduri de identificare a modului
evoluției cererii implementate corespunzător.

\subsection{Tehnologii de comunicare}

Am menționat anterior că microserviciile ne cumpără flexibilitate, însă din acest motiv avem
și mai multe probleme întrucât mereu când alegem un tip de tehnologie trebuie să limităm
dezavantajele și în același timp să maximizăm avantajele pe care le poate aduce. Astfel putem să alegem
tehnologii mai vechi care au fost testate dar limitate sau tehnologii noi care promit că aduc îmbunătățiri
dar e posibil să nu fie lipsite de erori ce s-ar rezolva în timp.

Când alegem o tehnologie, trebuie să avem grijă că aceasta trebuie să fie compatibilită cu
serviciile pe care le avem, de asemenea trebuie să nu ne limiteze în alegerea tehnologiilor viitoare.
În acest scop modul de comunicare al microserviciilor ar trebui să fie clar definit întrucât 
nu putem fi siguri ce elemente sunt folosite și care nu. De asemenea tehnologia aleasă
nu ar trebui să inhibe comunicarea cu alte microservicii.

Remote procedural call (RPC) (Apel procedural la distanță) este un mod de apelare de funcții local
însă care se execută într-un serviciu la distanță. Pentru a face asta posibil sunt necesare 
definirea unor scheme explicite precum SOAP sau gRPC definite ca limbaje de definit interfețe
(interface definition language). Definirea unor scheme explicite face ca folosirea tehnologiilor
diferite să fie mai ușoară însă dacă nu vrem să folosim astfel de scheme putem folosi Java RMI
care necesită folosirea aceleași tehnologii însă fără schemă definită. Folosind o astfel de tehnologie
ne permite generare de cod pe care clientul le poate folosi pentru a comunica cu serverul 
făcând implementarea pe partea clientului mai ușoară întrucât nu trebuie să facă ceva în mod special.
Limitările acestei tehnologii reprezintă restricționarea tehnologiilor folosite (mai ales în cazul
Java RMI) însă celelalte menționate funcționează cu diferite tehnologii. Ascunderea faptului
că prin aceste metode se face un apel peste Internet poate fi înșelător și ascunde complexitatea.
De asemenea atunci când producem modificări asupra codului serverului trebuie să generăm codul
pentru client, însă dacă adăugăm sau eliminăm câmpuri din interiorul funcțiilor, clientul trebuie 
să facă mai multe modificări deși el poate să nu folosească acele părți din cod. Acest tip de abordare
este potrivit pentru comunicarea blocantă între doua sisteme asupra căruia știm când se modifică lucrurile și 
în ce fel, astfel cunoaștem cât de multe modificări vor trebui să fie făcute.

Representational State Transfer (REST) prin HTTP, în acest tip de arhitectură avem definite 
niște resurse ce pot fi accesate prin link-uri sub diferite forme ce permite identificarea și
modificarea acestora în funcție de modul de accesare. Acest tip de abordare funcționează 
foarte bine cu verbele HTTP predefinite (GET, POST, UPDATE) întrucât știm la ce comportament 
ne putem aștepta indiferent de resursa pe care o accesăm. Faptul că folosim HTTP ne permite să
folosim toate elementele dezvoltate pentru a îmbunătății sistemul precum caching, echilibrări de trafic,
unelte de monitorizare existente dar și elemente de securitate precum autentificare și certificate.
O altra trăsătură adusă în REST este Hypermedia as the engine of application state (HATEOS) ce permite
să nu cunoaștem întreg URI-ul către resursă ci doar cel de început. O dată intrat pe prima intrare,
aceasta va afișa alte intrari pe care putem intra și ce am putea accesa dacă intrăm pe ele. Acest lucru
este util pentru că nu trebuie să definim niște căi explicite ci acestea pot fi căutate însă
pentru a face asta cresc numărul de cereri pe care trebuie să le faci doar pentru a executa o
singură comandă. Atunci când folosim această tehnologie nu avem o modalitate simplă de a crea
librării pe care clienții serverului le pot folosi, din acest motiv s-a creat specificația
OpenAPI din Swagger ce permite ca din anumite informații să generezi automat o documentație 
dar și cod pentru o varietate de limbaje însă nu a fost adoptat la fel de mult întrucât 
ar fi dificil de implementat pentru proiectele existente. Prin acest format putem trimite o
varietate de formate de date însă din cauza cerințelor adăugate de HTTP atunci când creăm o cerere
ar putea să împiedice performanțe ridicate atunci când nu avem date multe de trimis. Putem folosi
acest mod de comunicare cam în toate situațiile în care avem nevoie de comunicare unu la unu între
servicii, fiind un stil comun de comunicare este suportat de majoritatea tehnologiilor 
deci ne garantează ca am avea compatibilitate.  

Ambele tipuri de apeluri au aceași problemă, în unele cazuri ar trebui să facem mai multe 
apeluri pentru aceași resursă întrucât nu avem suficiente date, sau în cazul în care un 
câmp dintr-o cerere se schimbă acesta ne-ar forța să modificăm în mai multe locuri. Astfel de probleme
sunt rezolvate de GraphQL ce permite să specificăm exact ce câmpuri avem nevoie fără a fi nevoie
să facem mai multe cereri. Deși ar ușura cererile la nivelul clientului, dezavantajele apar 
la server, în primul rând acesta este un Query Language, deci dacă cererea este complicată
atunci server-ul va trebui să o proceseze, dacă se adună multe cereri de tipul acesta
ar scădea performanța. De asemenea, întrucât de fiecare dată se face un query, caching-ul 
este mai dificil de realizat și nu poate fi folosit la fel de eficient pentru a scri informații.

În arhitectura bazată pe evenimente putem folosi agenți de mesaje (Message Brokers), 
aceștia funcționează ca un intermediar pentru comunicațiile între microservicii, astfel în loc
să comunice direct acestea comunică prin intermediar trimițând orice tip de mesaje.
Acestea oferă diferite facilități în modul în care trimit mesaje, printr-o coadă în care
se pun mesajele iar consumatorii doar preiau direct din coada aceasta sau prin topică ce permite
consumatorilor să se aboneze la un anumit tip de mesaj iar atunci când un mesaj vine ce îndeplinește
criteriul, acesta va fi trimis la toți abonații. Principalul motiv pentru care am folosi o astfel
de tehnologie este pentru că oferă diferite caracteristici care ar fi greu de implementat
precum garanția livrării, astfel dacă nu există nici-un microserviciu care să accepte un mesaj,
atunci el va fi livrat la primul care va fi disponibil să prelucreze acel mesaj. Alte caracteristici
includ garantarea ordinei în care mesajele sunt primite sau să îți permită să trimiți același 
mesaj către mai multe topice.

Asupra acestor diferite moduri de comunicare apar diferite moduri de a trimite un mesaj, 
referindu-ne la compoziția acestora. Astfel, acestea pot fi sub formă de text sau sub formă
binară. În general, putem trimite sub formă text adică JSON sau XML însă dacă o să ne intereseze
să eficientizăm viteza atunci va fi necesar să encodăm mesajele trimise și să le trimitem sub
formă binară însă acesta nu este un proces ce ar aduce suficient de multe îmbunătățiri.

Atunci când am vorbit de comunicarea între microservicii am menționat importanța creării
unei interfețe sau a unei scheme pentru a afișa modul cum comunică microsericiul cu exteriorul.
Acest lucru permite consumatorilor să știe la ce să se aștepte și să implementeze mult mai 
ușor crearea legăturii iar pentru cei ce administrează microserviciul ar cunoaște ce 
câmpuri sunt expuse și ce trebuie să facă. Deși unele tehnologii necesită prezența unei
scheme sub o formă sau alta, schemele ne ajută și în crearea unei documentații consistente
ce ajută la implementare. Cu ajutorul unei scheme putem găsi probleme mult mai ușor, asemănător
cu modul în care funcționează limbajele de programare ce necesită specficarea unui tip
atunci când declarăm variabile și limbajele interpretate ce pot avea orice valoare, 
avem un nivel în plus de siguranță, cel la nivelul la compilare, astfel într-un limbaj
în care acest tip de verificare ar lipsii, unele probleme apar doar în producție. Atunci când
trebuie să modificăm o schemă aceasta se poate face în doua feluri, structural când câmpurile se schimbă,
astfel clienții trebuie sunt nevoiți să administreze noul câmp, sau semantice atunci când câmpul
nu își schimbă numele însă modul de calculare este diferit. Acest ultim tip este destul de periculos
întrucât este ușor de ignorat dacă nu apar probleme evidente. În general este recomandat să 
creăm scheme de tipul acesta, mai ales dacă microserviciile sunt deținute de echipe diferite, 
în cazul acesta nivelul de comunicare dintre cele două este îmbunătățit iar problemele sunt
rezolvate mult mai ușor.

\subsection{Schimbări în funcționalitate}

Atunci când este vorba de microservicii, este inevitabil ca funcționalitatea acestora să
se schimbe la un moment dat, iar în cazul cel mai fericit în care cuplarea și coeziunea sunt favorabile,
atunci cel mai probabil nu trebuie să modificăm alte microservicii, însă de multe ori acesta nu este
cazul.

O primă parte sunt metodele de evitare ale acestor probleme, cel mai simplu prin asigurarea
că funcționalitatea microserviciului nu este schimbată ci doar se adaugă funcționalitate. De asemenea,
putem implementa în consumatori ca aceștia să fie mai toleranți și să încerce să își găsească 
datele de care au nevoie chiar dacă se schimbă schema, acest lucru se poate face și prin prelucrarea
strictă a câmpurilor de care avem nevoie și nu de întreg fișierul. Putem folosi tehnologii care să
permită adaptare ușoară, de exemplu atunci când serializăm un obiect in Java acesta are o anumită
versiune, însă dacă modificăm obiectul adăugând un câmp această versiune s-ar schimba deși 
câmpurile de care am avea nevoie ar rămâne la fel, acest lucru creând o incompatibilitate, însă
dacă am folosi ceva asemănător cu HATEOS ar fi mult mai ușor de adaptat fără nicio modificare.
Este important ca orice modificare adusă datorită unei schimbări în funcționalitatea unui microserviciu
să fie prinsă cât mai rapid, de asta sunt importante testele iar cu ajutorul schemelor definite 
explicit putem folosi diferite unelte ce pot valida diferențe între modificări la nivelul 
protocoalelor, schemelor sau a specificiațiilor OpenAPI.

Însă uneori este imposibil să mitigăm toate problemele și este nevoie să adăugăm schimbări pe care
și alte microservicii sunt obligate să le facă, astfel apar dificultăți în orchestrarea ordinii acestor schimbări.
Una dintre soluții este anunțarea schimbărilor pe care vrem să le facem și să lăsăm toți consumatorii 
să lanseze versiuni noi ce tolerează schimbările și după aceea să lansăm varianta schimbată.
Acest lucru nu este în general suficient întrucât datorită modului de planificare aceste 
apariția acestor schimbări în consumatori poate să fie destul de lentă. O altă metodă 
de administrare a schimbărilor este de a lansa două servicii în paralel, una ce conține 
modificările și una care nu le conține pentru microserviciile care încă nu s-au adaptat.
Acest lucru este de evitat din același motiv ca precedentul dar și faptul că crează multe 
dificultăți să ai mai multe servere de același timp lansate. De asemenea se poate păstra
versiunea veche și doar crearea unei noi versiuni a endpoint-ului. Versiunile mai vechi
pot fi schimbate să fie compatibile cu versiunea nouă și eventual retrase când suntem siguri că
toți clienții au primit modificările necesare. Acest lucru se poate face prin monitorizarea traficului
pe interfață. Acesta este tipul preferat de adus schimbări întrucât este cel mai convenabil pentru
consumatori dar și pentru server.
