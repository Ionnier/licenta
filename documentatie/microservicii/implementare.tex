Trecerea de la o aplicație monolitică la una bazată pe microservicii uneori nu este
ușoară. Unul din principalele motive este o posibilitate de a regândii modul cum accesăm
anumite elemente. Atunci când totul se află în cadrul unui singur proces, acessarea datelor
devine mai grea întrucât nu avem acces la ele și uneori nu este rentabil să căutăm după ele.
Atunci când apelăm o funcție local, timpul necesar pentru executarea acesteia este neglijabil,
atunci când această funcție trebuie să apeleze un alt microserviciu, lucrurile se complică și
trebuie decis dacă chiar avem nevoie. Un apel făcut printr-o rețea mereu va avea o întârziere, de asemenea
cresc locurile în care acesta poate eșua. În dezvoltarea aplicațiilor mobile întrucât 
ne așteptam ca aplicația să reacționeze la atingeri, nu putem să facem anumite operații pe
thread-ul principal, acestea pot fi apelarea lucrurilor de pe Internet sau accesarea unei baze
de date (chiar și dacă este de tipul SQLite3 fiind locală), aceasta pune în perspectivă 
impactul pe care acest tip de apeluri le pot avea.

Atunci când modificăm modul în care comunică un microserviciu trebuie să avem grijă la modul
cum ar afecta alte microservicii, sau atunci când acest lucru este necesar trebuie să orchestrăm
aplicarea acestor schimbări. Un alt concept ce într-o arhitectură monolitică se tratează ușor
este tratarea erorilor, întrucât toate se întâmplă în același mediu de lucru putem avea o viziune
asupra ce poate să cadă și în ce mod. Într-un sistem distribuit, pot apărea mai multe tipuri
de erori care se datorează rețelor, de exemplu crash-uri de microservicii ce necesită resetări,
erori de omisie de mesaje, când încercăm să apelăm un microserviciu însă acesta nu trimite un răspund,
în același timp acest răspund poate să fie mult prea rapid și să nu putem să îl recepționăm
sau prea târziu și să considerăm că nu am primit, dar și erori arbitrare precum unele în care 
ne așteptăm la un răspund și nu primim ce avem nevoie.

Toate aceste aspecte trebuie tratate atunci când lucrăm cu microservicii, însă fiind vorba
de o arhitectură paralelă ce oferă independență avem flexibilitatea de a alege ce folosim și ce nu.
De aceea este important să alegem tehnologii și stiluri de comunicare care oferă 
performanțele dorite încercând să minizăm timpul de răspuns și tipurile de avarii ce pot apărea.

\subsection{Tipuri de comunicare}