Testarea este o parte importantă în dezvoltarea produsului, avantajele creării de produse
sustenabile ce nu conțin defecte sunt destul de evidente întrucât clienții pot aprecia
faptul că aplicația nu se strică și pot construi o reputație în acest sens. Pe de altă parte,
apariția unei probleme majore în produs poate fi extrem de costisitoare în funcție de momentul
în care aceasta este găsită, astfel este important ca în procesul creării defectele să
fie prinse și rezolvate din timp.

Testarea poate să aibă o reputație scăzută fiind asociată ca ceva repetitiv ce trebuie făcut
manual însă aceasta a avansat și se face în diferite moduri inclusiv automat astfel poate
să fie la același nivel cu dezvoltarea însă necesitând un alt set de abilități și un mod
de gândire diferit.

Însă acest tip de testare se referă la oamenii dedicați pentru asigurarea calității produsului,
deci echipa de QA. Însă ei nu sunt singurii cu responsabilitatea de a monitoriza
calitatea sistemului, acesta pornește direct de la dezvoltator. În general, un tester poate
să nu aibă acces sau să nu se uite la cod, numit si black-box testing, în timp ce primind
acces se numește white-box testing iar atunci când avem acces parțial, gray-box testing.

Testele pot urmării lucruri diferite, astfel conform cărții „Agile Testing” de Lisa Crispin
și Janet Gregory putem împărții testele în funcție de ce încearcă să descopere. Testele de acceptare
testează dacă funcționalitatea este conform specificațiilor. Testarea exploratorie se face
în general manual și încearcă să verifice dacă există probleme în implementare. Testele unitare
sunt automate și verifică dacă nu s-a stricat funcționalitatea funcțiilor în urma unor modificări noi.
Testele de proprietate verifică performanța sistemului și capacitatea acestuia, se face
prin folosirea de unelte specializate și urmărește timpul de răspus, scalabilitatea sau reziliența.

Putem grupa aceste categorii și să obținem mai multe clasficări, testele de afaceri ajută
alte persoane să înțeleagă calitatea produsului (TU, TE) și testează o parte mai largă din produs,
testele ce vizează tehnologia sunt mai rapide și au componente care sunt automatizate (TU, TP).
Pentru a observa calitatea produselor se pot folosi teste care au ca scop asistența în dezvoltarea
pentru a nu apărea regresii (TU, TA) dar și unele care critică produsul în funcție de performanță
sau adaptibilitate (TE, TP). În general testele se fac în timpul dezvoltării, însă există teste 
care se fac pe produsul care este în piață.

Testele de serviciu au ca scop testarea unui microserviciu, acestea evită testarea UI-ului
ci doar a componentelor folosite de UI ca să aducă funcționalitate utilizatorului. Limitând
testele la un singur microserviciu, testăm capacitatea acestuia iar orice comunicare cu exteriorul
este mimată, astfel cauzele de eșec pot să fie strict legate de microserviciul testat
Aceste tipuri de teste, întrucât nu testează componenta grafică pot sa fie la fel de rapide
ca testele unitare însă întrucât pot intra în componență apeluri la baza de date sau la rețea.
Astfel în implementarea acestora există necesitatea înlocuirii comunicării cu exteriorul și se
face în general fie prin mimicarea datelor sau prin înlocuirea acestora. Mimicarea se referă
că la orice tip de apel al unei funcții, indiferent de datele de intrare, timpul la care a fost
făcut sau numărul de ori de câte este apelată acesta va returna un singur set de date. Înlocuirea
se face prin replicarea funcționalității astfel putem apela funcții care procesează datele de 
intrare și returnează date în funcție de acestea. Putem să dezvoltăm această idee și să 
încercăm să creăm servere de test care furnizează această înlocuire existând unelte pentru aceasta,
de exemplu mountebank ce poate să mimice date trimite prin HTTP/S, TCP sau SMTP însă nu poate să
mimice comunicarea prin evenimente.

Însă chiar dacă facem aceste teste, nu putem să garantăm că putem livra microserviciul.
Testele end-to-end se fac prin mimicarea utilizatorului și a diferitelor comportamente pe care 
acesta le-ar face, astfel verificând funcționalitatea sistemului prin interfața cu care el comunică. 
Acestea pot să implice multiple microservicii ce trebuie să nu aibă probleme, însă având o arie mai mare 
de acoperirea acestea sunt mai greu de testat iar atunci când pică nu se poate observa din 
prima posibila cauză. Una dintre problemele care arpare în aceste teste este nevoia de a rula
mai multe microservicii in același timp, astfel pentru fiecare din aceste microserviciu s-ar rula aceleași
teste. Acesta ar duce la o repetare de teste care nu s-au schimbat, astfel o metodă de a eficientiza
procesul este de a combina testele end-to-end pentru mai multe microservicii in același timp.

Aceste tipuri de teste acoperă o arie extinsă de cod iar fiecare dintre acestea poate să aibă
probleme, astfel pot exista în suita de teste unele teste care uneori pică deși uneori trec, cauza
unor astfel de situații poate să fie probleme temporare de rețea ce cauzează un timeout.
În astfel de cazuri ar trebui ca acest test să fie fixat, fie eliminat întrucât existența 
unui test neconcluziv poate duce la ignorarea acelui test, astfel este ca și cum acesta nu ar
rula de la bun început. 

Întrucât testele end-to-end acoperă mai multe microservicii apare problema asocierii 
responsabilității scrierii acestora. Întrucât în același timp mai multe echipe sunt implicate,
acestea toate ar avea o parte de responsabilitate însă scrierea unui astfel de test 
include cunoașterea sistemului întreg și observarea locurilor ce pot fi mai sensibile,
astfel companiile mari pot investii în crearea de echipe dedicate ce au ca rol scrierea, 
rularea și rezolvarea (când este cazul) a testelor end-to-end. Însă crearea unor teste
end-to-end complexe poate duce la timpuri mari de executare ale acestora, astfel 
feedback-ul compatibilității întârzie iar în acest timp se pot aduna mai multe caracteristici
adăugate de dezvoltatori, astfel se încalcă regula menținerii o prioritate a rezolvarilor
problemelor ce sunt asociate pipeline-urilor, în acest caz putem analiza suita de teste și să
păstrăm doar esențialele, mai ales când un build poate să pornească suite de teste end-to-end diferite
întrucât acesta este prezent în mai multe procese care se desfășoară în cadrul mai multor
microservicii.

Problemele cu testele end-to-end duc la crearea unor probleme asociate cu arhitectura paralelă,
întrucât acestea ar trebui să ruleze separat, atunci ar putea să fie și testabile independent,
astfel putem rula teste end-to-end mai rar dacă avem diferite caracteristici care să 
ajute la interoperabilitatea precum interfețele explicite sau prin folosirea unor unelte precum Pact 
putem să creăm o schemă a comunicațiilor exterioare ce include așteptările pe care le avem de la 
aceștia. Acestea pot fi rulate după de dezvoltatorii microserviciului și să cunoască 
dacă apar probleme de compatibilitate. Acestea sunt folositoare întrucât nu trebuie să mai rulăm
mai multe microservicii în același timp ci rulează ca testele unitare sau cele de serviciu în cadrul
unui singur microserviciu, astfel devin mult mai rapide.

Majoritatea testelor rulează în timpul perioadei de dezvoltare, însă putem rula teste 
și pe produsul final. Înainte de lansare se poate lansa o suită de teste numită Smoke, ce 
are ca scop verificarea funcțiilor de bază, de multe ori aceasta ar trebui să ruleze înainte de testarea
manuală întrucât căderea acestuia ar însemna lipsa funcționalităților de care am avea nevoie 
în mod normal, însă ca o ultimă măsură putem să îl rulăm alături de un test de Sanity ce conține
o gamă mai largă de de teste înainte unei posibile lansări. De asemenea, alte
teste în producție pot fi cele de A/B sau feature toggles ce includ lansarea unei funcționalități
în două variante pentru a fi mai ușor de decis care dintre variante putem să o alegem (un caz 
recent a fost unul făcut de eMAG în care au testat performanța vânzărilor încercând 
să ascundă recenzille asupra acestora) sau Canary releases în care utilizatorii se pot 
întregistra pentru a primi varianta mai nouă a produsului, aceștia sunt limitați iar în cazul 
în care apar erori, doar o parte mică din utilizatori a fost afectată. O rulare în paralel 
se referă la lansarea unei funcționalități și păstrarea versiunii anterioare, orice cerere este
trecută prin alte servicii și sunt evaluate anumite metrici. Ingineria haosului se referă la adăugarea 
de greșeli intenționat în produs și testarea acestuia dacă poate să își revină. Acestea
pot fi considerate și metode de lansări de produs. Alte tipuri de teste pot fi cele care simulează
un utilizator normal în timp ce produsul este în piață pentru a verifica dacă funcționalitățile rămân 
la fel și nu există probleme de configurări ale mediilor de lucru.

În cazul apariției a unei probleme în producție, unele companii rulează simulări și rapoarte
a modului de lucru în cazul apariției unei probleme. În cazul acestora se pot definii concepte 
precum timpul mediu de reparare(MTTR) sau timpul mediu între erori(MTBF). Prima dintre ele
se poate rezolva ușor prin crearea unei restaurări a versiunii anterioare iar a doua prin 
îmbunătățirea capacității sistemului de a rezista la erori sau de a le trata. Aceste două metrici
sunt la fel de importante ca și testarea întrucât ambele au același scop, asigurare calității 
unui produs.

Majoritatea testelor rulate vor avea ca scop testarea caracteristicilor și validarea
că acestea funcționează în parametrii normali, însă datorită volatilității traficului
pot apărea scenarii complexe în care pot apărea un influx de trafic, astfel există 
dorința de a testa după funcționalitate, adica aspecte precum performanța sistemului.
În cadrul unei arhitecturi paralele aceasta poate cauza probleme întrucât într-o aplicație
monolitică comunicarea se face într-un singur loc, însă datorită necesității a comunicării 
colective pot apărea momente în care sunt necesare crearea mai multor cereri, acestea pot
cazuri în care pentru microservicii suprasolicitate să creze creșterea latenței sau
chiar acestea să nu mai poată fi accesibile. Astfel pot exista mai multe teste de performanță,
cele de enduranță ce testează sistemul pe o perioadă lungă de trafic dar în parametrii normali,
de stres în care testăm sistemul ăn afara parametrilor normali, ale spike-urilor ce sunt creșteri
mari in trafic pe perioade scurte de timp și de scalabilitate ce observă capacitatea 
sistemului de a servi atunci când numărul de utilizatori crește. Toate acestea ar trebui să 
se facă cunoscând funcționalitatea normală a sistemului pe trafic real, în absența acestei condiții
putem să ne concentrăm pe direcții pe care utilizatorul nu le ia. Într-o arhitectură paralelă,
la fel ca în securitate, capacitatea acestuia de a rula eficient este dată de capacitatea celei
mai încete piese astfel putem crea scenarii de test în care testăm lipsa unor componente ce
au ca scop creșterea rezilienței în cazul erorilor.

