Indiferent de platforma sau tipul dezvoltării pe care îl facem, front-end, back-end sau IoT,  
securitatea va fi una din problemele care vor fi ridicate atunci când dezvoltăm produsul.
Însă aprofundarea acestei părți nu este ușoară întrucât necesită o mentalitate specială în 
care luăm în considerare un adversar. În general, în timpul dezvoltării construim elemente 
și le construim într-un mod în care să funcționeze eficient, însă există persoane care 
vor să abuzeze de vulnebaribilățile ce pot apărea la orice nivel al aplicației.

Din acest motiv, în cadrul unei companii există persoane speciale care se ocupă de securitate.
Aceștia cercetează și studiază posibilele vulnerabilități care apar și crează anumite investigații
asupra produsele companiilor. În același sens, se pot folosi și unelte speciale în interiorul pipe-ului
care au ca scop scanarea fișierelor și raportarea posibilelor vulnerabilități, chiar dacă nu este amănunțit
sau o scanare în detaliu, aceasta ar oferi feedback rapid ce poate fi soluționat în același mod.

Prin stilul lor, microserviciile ne oferă posibilitatea de a îmbunătății securitatea sistemului
întrucât fragmentarea duce la ascunderea informației către serviciile din exterior, astfel un
atacator ar primi de fiecare dată un minim de informație, însă pe cealaltă parte, microserviciile sunt
foarte complexe și necesită mult mai multe setări, pe lângă infrastructură și datele pe care 
acestea le păstrează, cresc numărul de rețele ce trebuie securizate, numărul de dependințe
ce pot proveni de la terți (Eventbus, load-balancers, log tools) cresc și trebuie scanate
regulat pentru vulnerabilități, astfel numărul de sisteme ce trebuie să fie luate în considerare 
atunci când construim metodele de protejare ale sistemului.

