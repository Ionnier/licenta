Indiferent de platforma sau tipul dezvoltării pe care îl facem, front-end, back-end sau IoT,  
securitatea va fi una din problemele care vor fi ridicate atunci când dezvoltăm produsul.
Însă aprofundarea acestei părți nu este ușoară întrucât necesită o mentalitate specială în 
care luăm în considerare un adversar. În general, în timpul dezvoltării construim elemente 
și le construim într-un mod în care să funcționeze eficient, însă există persoane care 
vor să abuzeze de vulnebaribilățile ce pot apărea la orice nivel al aplicației.

Din acest motiv, în cadrul unei companii există persoane speciale care se ocupă de securitate.
Aceștia cercetează și studiază posibilele vulnerabilități care apar și crează anumite investigații
asupra produsele companiilor. În același sens, se pot folosi și unelte speciale în interiorul pipe-ului
care au ca scop scanarea fișierelor și raportarea posibilelor vulnerabilități, chiar dacă nu este amănunțit
sau o scanare în detaliu, aceasta ar oferi feedback rapid ce poate fi soluționat în același mod.

Prin stilul lor, microserviciile ne oferă posibilitatea de a îmbunătății securitatea sistemului
întrucât fragmentarea duce la ascunderea informației către serviciile din exterior, astfel un
atacator ar primi de fiecare dată un minim de informație, însă pe cealaltă parte, microserviciile sunt
foarte complexe și necesită mult mai multe setări, pe lângă infrastructură și datele pe care 
acestea le păstrează, cresc numărul de rețele ce trebuie securizate, numărul de dependințe
ce pot proveni de la terți (Eventbus, load-balancers, log tools) cresc și trebuie scanate
regulat pentru vulnerabilități, astfel numărul de sisteme ce trebuie să fie luate în considerare 
atunci când construim metodele de protejare ale sistemului.

La baza securității pot sta anumite gânduri ce au ca scop restricționarea și controlarea
accesului. Anumite elemente le-am menționat până să ajungem aici, principiul ascunderii
al informației se referea la limitarea numărului de date pe care le expunem, în cadrul
securității ne putem gândii la limitarea privilegilor, astfel, ar trebui să oferim
drepturi fiecărui microserviciu cu care comunicăm numai cât are nevoie pentru a funcționa normal,
dacă el folosește doar câteva funcționalități atunci acesta ar trebui să aibă vizibilitate
numai asupra lor, motivul fiind, dacă acesta este compromis atunci atacatorul nu poate să
acceseze mai mult decât microserviciul avea nevoie. Atunci când construim o soluție de securitate
pentru sistemul nostru, este indicat să conțină mai multe nivele și moduri ce oferă protecție.
Există atacuri ce s-au produs iar în timpul lor nu a existat vreo metodă de alertare.
Astfel, putem să facem distincția între modurile preventive de securizare, acestea au scopul
de a preveni apariția unui atac, iar dacă se întâmplă să se limiteze daunele ce pot fi produse,
printre acestea includ acțiuni precum securizarea variabilelor secrete, crearea unui sistem
de autentificare și criptarea datelor. Putem spori detecția prin setarea aplicațiilor de firewall
sau alerte atunci când se accesează anumite date. Pentru a mitiga atacul putem folosi
diferite procedee însă cel mai important este comunicarea eficientă între echipe iar ulterior
metode de a recupera sistemul. Implementarea automatizării este o metodă de a crește securitatea
întrucât forțează gândirea metodelor de acces dar și eliminarea greșelilor umane iar
adăugarea unor metode de analizare a securității în pipeline ce pot investiga erorile simple
la nivel de cod.

Tot de la bazele securității provine crearea unui model de amenințare. Procesul creării
acestuia include mai multe etape ce au ca scop cunoașterea sistemului și identificarea atacatorilor.
Prima etapă se referă la identificarea elementelor de valoare din componența sistemului, modul în care
acestea ar putea fi abuzate și identificarea posibililor atacatori ce ar putea apărea pentru a
cauza probleme în interiorul sistemului. În urma acestuia, putem să ne gândim la modul în care
o să protejăm datele și accesul la acestea. Întrucât nu putem preveni inevitabilul
de fiecare dată următorul pas este creșterea metodelor de detecție, în general prin folosirea
de unelte dedicate la nivel de rețea și infrastructură. Capacitatea de răspuns în fața
unui atac nu se referă doar la metodele de oprire ale atacului dar și alte elemente precum
comunicarea problemelor către clienți și parteneri. Metodele de recuperare se referă la
revenirea la normal și aplicarea elementeleor descoperite în cadrul atacului pentru a preveni
astfel de incidente.

Întrucât dezvoltarea pornește de la aplicație, de aici putem începe și cu crearea de
practici bune pentru a crește securitatea sistemului. În general, o aplicație va avea
cel puțin câteva date sensibile ce au de a face cu autentificarea, fie a microserviciilui în sine
către baze de date sau alte microservicii sau alte utilizatorilor. Întrucât numărul de conexiuni
poate fi ridicat, am putea încerca crearea unor date generale și distribuirea acestora însă ar
încălca unul din princiipile menționate anterior. Microserviciile sunt create pentru a oferi
un serviciu către utilizatori, în general aceștia pot să nu fie experimentați și să ignore
sfaturi precum folosirea parolelor unice, lungi și administrare de programe speciale, iar
în cazul de față se poate ajunge ca dauna provocată să nu se producă doar în cadrul
seriviciului nostru ci prin refolosirea datelor chiar și către alte servicii pe care nu le administrăm.
Pe lângă sfaturile simple precum securizarea parolele prin folosirea salt-ului și a algoritmilor
siguri într-un timp ridicat (pentru a preveni atacurile de tip brute force) și folosirea de 2FA, putem extrage
din articolul \cite{troy_hunt2017} al lui Troy Hunt în care a prelucrat recomandări de la NIST.
Poate exista o limită la dimensiunea parolelor însă restricția ar trebui să fie mai mare
de 64 de caractere suportând și caracterele UNICODE nerestricționate de reguli precum
o anumită compoziție permițând folosirea de unelte de memorare a unor parole ce nu exipră
ce sunt verificate pentru a preveni prezența acestora într-o bază de date de parolă existente
și oferirea utilizatorului unei pagini în care își poate observa activitățile și dispozitivele.
Pe lângă acestea, pot exista secrete, adică date senzitive care au ca scop funcționarea
microserviciului într-un anumit mediu, acestea pot fi perechi de chei asimetrice publice/private
dar și de API. Diferența dintre acestea și parolele sunt faptul că noi administrăm modul lor
de viață, controlând crearea, distribuirea, stocarea, monitorizarea și schimbările. În acest context,
în funcție de modul deployment-ului putem folosi caracteisticile acestora, Kubernetes prezintă
o formă simplificată de secret management, acestea sunt memorate în memorie RAM și definite prin
perechi chei-valoare iar în interiorul pod-ului sunt accesate sub forma de fișiere montate local,
limitarea lor este că orice schimbare a secretului nu va avea efect decât atunci când pod-ul este
restartat sau putem folosi platforme specializate precum Hashicorp Vault ce oferă o soluție
complexă în acest domeniu. Implementând o metodă de rotație ne permite să generăm chei atunci
când avem nevoie pentru o perioadă scurtă de timp, astfel dacă acesta este compromisă
atunci putem limita pagubele ce pot fi produse, în același timp revocarea ne ajută
în cazul conștientizării unei expuneri iar acestea ar trebui să ofere acces doar la
resursele de care avem nevoie.

Securitatea aplicației nu se referă strict la ea ci și la modul cum interacționăm cu ea,
adică prin infrastructură. Trebuie să fim la zi cu actualizările sistemului pe care
aplicația rulează, dacă folosim infrastructură din cloud probabil acestea sunt făcute deja.
Alte servicii pe care operatorii cloud le oferă sunt opțiunile de backup și encriptare a
datelor la idle. Backup-urile ne permit să ne întoarcem la versiuni anterioare ale aplicației,
astfel nu avem nevoie de backup-ul unei mașini propriu zise ci doar a modului de a ajunge în aceași stare.
Codul și imaginile proceselor sunt deja track-uite prin VCS, folosind infrastructură ca și cod sau
GitOps atunci avem și configurarea sistemului, rămând doar datele utilizatorilor ce în general
se fac prin baze de date ce probabil vin cu versiunile proprii de backup în funcție de
tehnologia aleasă. Pentru ca back-up-urile să fie eficiente, trebuie să avem încredere că acestea
funcționează astfel putem să creăm un proces în care frecvent facem restaurare a backup-ului
pentru a fi testat.

Microserviciile funcționează prin comunicarea la distanță prin diferite rețele, astfel se disting două
cazuri, întrucât microserviciile sunt lansate într-un loc în care avem control total și
putem identifica orice neregularitate, putem avea încredere deplină că atunci când vine
o cerere din interior, aceasta vine de la un serviciu de încredere. În partea opusă apare
conceptul de zero încredere, în care toate serviciile care încearcă să ne acceseze sunt deja
compromise, în acest caz trebuie să luăm precauții. Acesta este un mod de gândire specific
securității, astfel cererile trebuie inspectate iar datele ce sunt comunicate trebuie să fie criptate.
Atunci când construim un sistem, ar fi dicil ca acesta sa fie descris ca cel din ultimul caz,
iar siguranța ar putea fi afectată dacă ar fi ca în cel din primul, astfel putem să le
folosim pe amândouă în funcție de datele pe care le folosim.

Microserviciile vor face mereu schimb de date, fie ele către un singur alt microserviciu,
fie către mai multe. Din acest motiv am vrea ca aceste date să nu fie interceptate
de către un terț. Dacă comunicarea este făcută prin HTTP putem folosi TLS pentru a securiza traficul,
iar dacă comunicarea se face printr-o aplicație de comunicare de evenimente atunci
putem consulta documentația platformei și să găsim modalități de securizare. Atunci când
dorim să securizăm o comunicare, sunt câteva puncte pe care le urmăm. Una din ele este
stabilirea autenticității server-ului, astfel serviciul care face cererea vrea să știe dacă
cel către care face cererea este chiar cel căruia vrea să îi facă cererea. Astfel de cazuri
au făcut probleme pe Internet, astfel s-a dezvoltat HTTPS pe care îl putem folosi sub diferite
forme. În același sens, serverul ar vrea să cunoască dacă clientul este chiar cel care încearcă
să facă cererea. Astfel de cazuri nu prea există pe Internet-ul public întrucât serverele sunt accesibile,
iar autentificarea clientului nu este o prioritare, mai ales când utilizatorii folosesc
lucruri precum combinații de autentificare ce cuprind parole. În cazul microserviciilor
putem folosi protocoale precum Mutual TLS pentru a stabili autenticitatea clientului. Implementarea
acestuia ar fi fost dificilă în trecut fiind o problemă distribuirea certificatelor însă
datorită uneltelor precum Hashicorp Vault acestea devin mai ușoare. După ce participanții
la comunicare sunt autentificați aceștia își doresc ca mesajele lor să fie securizate.
În acest sens ne interesează dacă mesajul poate să fie văzut de un terț, însă folosind
HTTPS acesta va cripta datele folosind cheile publice iar cel care le decriptează va folosi cheia
sa privată, în același mod ne interesează dacă datele au fost modificate, pentru asta putem folosi
HMAC-ul prin calcularea un hash asupra datelor trimise pe care recipientul le poate calcula
pentru și să vadă dacă acestea au fost modificate.

După ce securizăm comunicarea, a doua prioritate este securizarea stocării, pentru asta putem folosi
o varietate de unelte, în general majoritate distribuitorilor de infrastructură cloud
oferă modalități de encriptare iar majoritatea bazelor de date oferă același lucru. Însă această
criptare poate deveni costisitoare, și din privința procesorului ce ar trebui constant să lucreze
dar și a spațiului de stocare necesare, din acest motiv ar trebui să ne limităm datele salvate
și să encriptăm orice dată primim și să decriptăm doar atunci când fără aceasta nu am putea să continuăm
operațiunile.

O altă problemă a microserviciilor este modul de autentificare și autorizare. Într-o aplicație monolitică,
aceasta nu era o problemă la fel de mare întrucât o dată autentificat toate cererile erau rezolvate
de aceași entitate, însă în cadrul unei arhitecturi decuplate acest lucru devine mai delicat.
Prima necesitate de autentificare este cea a microserviciilor, pentru a valida componentele
ce participă la comunicare, în acest sens putem folosi cheile pentru mutual TLS sau
chei de API pentru a cripta cererea și server-ul ar valida că acel client chiar există.
Însă dificultatea mai mare este autentificarea utilizatorilor. Întrucât aceștia folosesc mai multe
microservicii, ar trebui să se autentifice către fiecare, iar acestea ar primi informația despre utilizator.
Însă acest lucru este un inconvenient pe care vrem să îl evităm. Soluția ar fi crearea unui loc
central în care utilizatorul se autentifică iar după cererea este redirecționată către cererea
inițială căreia îi lipsea autorizația. Acest lucru este întâlnit la companiile ce oferă mai multe
servicii precum Microsoft la care avem Outlook, OneDrive, Teams, acestea au o pagină comună
de conectare, iar conectarea la una îți oferă posibilitatea de a te conecta la toate celelalte
cu ușurință. Acest sistem ar funcția prin intermediul unui intermediar care ar căuta dacă
utilizatorul este autentificat și dacă cererea pe care încearcă să o facă necesită autentificare și
este fie lăsat să își termine cererea fie către pagina de autentificare. O astfel de abordare
ar avea probleme întrucât doar prima cererea ar primi informații despre utilizator, iar
următoarele ar trebui să aibă încredere că microserviciul cu care comunică nu este compromis.
Pentru a rezolva această problemă am putea încerca să facem autentificarea în cadrul fiecărui microserviciu,
însă ar trebui să reutilizăm cod și deși am putea folosi librării nu ar fi sustenabil. O metodă
de autentificare poate fi folosira de JSON Web Tokens, aceștia pot fi semnați de către
sistemul de autentificare central, pasați către utilizator care de fiecare dată când va face un request
își va trimite token-ul, iar autentificarea de la un microserviciu la alt microserviciu s-ar putea
face cu ușurință întrucât ar consta doar în trimiterea token-ului. În componența token-ului
avem un header (ce conține informații despre algoritmul de semnare), datele și semnătura
propriu zisă. Deși JWT-urile sunt ușor de folosit, acestea nu sunt perfecte. O primă problemă
este setarea unui timp bun de expirare. Fiecare token are o dată de expirare setată de noi,
după trecerea acestui timp, Token-ul ar trebui să fie considerat invalid, astfel pot apărea
probleme în tranzacțiile de lungă durată, însă dacă timpul de expirare este mai ridicat
atunci dacă este compromis se pot provoca mai multe daune întrucât nu există o metodă ușoară
de a invalida un JWT. De asemenea, folosirea JWT-ului include administrarea informațiilor
necesare pentru criptare, adică cheile și timpul de expirare ce poate fi greu de sincronizat
între toate microserviciile care ar avea nevoie, însă unelte precum Vault ne poate ușura munca
în acest context.
