Partea de construire a executabilului este o parte importantă din dezvoltare. Pentru limbajele
compilate este un moment în care putem să vedem dacă avem erori ce în general sunt destul de minore
însă sunt prinse încă de la începutul rulării ceea ce ne permite să le rezolvăm rapid, având
feedback aproape instant de la compilator.

În aplicațiile monolitice, crearea executabilului este destul de ușoară întrucât există 
un singur produs ce se află într-un singur loc ce trebuie să fie lansat, 
astfel avem un număr de executabil minim. În contextul microserviciilor, 
fiecare microserviciu are „executabilul” propriu, însă lucrurile se complică 
atunci când luăm in considerare că fiecare microserviciu are și o bază de date locală pentru el,
astfel aceasta trebuie să vină la pachet și să fie configurată și executată.

Un concept devenit standard este integrarea continua („Continious Integration”) - CI, aceasta
presupune ca atunci când lucrăm într-o echipă, codul cu care lucrăm să fie mereu merge-uit
într-un loc central pe care se rulează teste automate pentru a verifica că nu există probleme
la integrare, iar dacă există atunci acestea ar trebui să fie principala prioritate. Acest 
concept își are baza în Agile și partea în care vrem ca mereu produsul să fie într-o stare
în care poate fi livrat. CI-ul presupune să avem artefacte create din codul ce este merge-uit,
astfel beneficiem de un istoric al build-urilor ce ne permite să testăm pe versiuni anterioare
în cazul în care apar probleme.

Necesarul pentru un sistem de CI eficient este să integrăm codul frecvent, ideal zilnic,
pentru a putea valida că codul rulează bine împreună, acest lucru făcându-se prin teste automate
pentru a asigura că modificările aduse nu aduc schimbări de comportament, fiind validat 
prin teste dar și corectarea codului ce face ca testele să nu funcționează, acest lucru este necesar
întrucât cu cât mai multe modificări aducem unui produs într-o stare stricată, va fi mai dificil 
de depanat.

Un mod de dezvoltare a caracteristicilor noi este de a crea branch-uri separate pentru fiecare.
Dacă cerința pe care încercăm să o dezvoltăm necesită mult timp, asta înseamnă că nu o să integrăm
codul decât la finalizare, ceea ce nu este în corcodanță cu principiul integrării frecvente.
Astfel, e mai indicat să fie un singur branch în care cei care dezvoltă adaugă schimbări frecvent
ceea ce evită problemele de integrare, frecvent numit și „trunk-based development”, însă nu este
perfect întrucât modificările făcute de alții sunt mai puțin vizibile, întrucât pull-request-urile sunt 
făcute ca o atenționare asupra modificărilor care vor apărea în cod, din acest motiv
problema cea mai mare este să așteptăm primirea acceptului la pull-request-urile lansate.

Integrarea continuă se face prin intermediul pipeline-urilor care sunt configurate prin 
intermediul anumitor unelte care sunt făcute special ca să îndeplinească aceste cerințe,
în cadrul lor se rulează mai multe etape în care se face o parte din procesul de construire al
artefactului. Acestea sunt definite cu ajutorul aplicației de CI. Cel mai important pas
este cel în care se rulează testele. În general este bine ca testele care rulează repede
să fie executate la început întrucât cu cât feedback-ul pipeline-ului este mai rapid, cu atât
fixarea lui va fi mai rapidă. În cadrul pipeline-ului se crează un artefact asupra căruia
se vor executa restul pașilor.

Un alt aspect de considerat este modul de organizare al codului sursă, când avem o aplicație
monolitică aceasta nu are motiv să fie despărțită în mai multe repo-uri, însă în cadrul microserviciilor
avem opțiuni.

Cea mai simplă alegere este să nu facem nimic special și doar să punem toate microserviciile
într-un singur director, iar la final avem un singur build. Din punctul de vedere al unui operator,
acesta trebuie să lanseze cod și să configureze un singur loc, iar un dezvoltator indiferent
de cât de multe microservicii ar schimba, acesta face un singur commit. Însă o astfel de arhitectură
face lucrurile simple la început dar complicate la final. Având un singur build, timpul pentru feedback
crește întrucât se pot rula teste și pentru microservicii care nu sunt schimbate. De asemenea,
nu știm ce microservicii ar trebui să fie lansate și care nu întrucât la final ce ne face
să lansăm totul în același timp. Astfel, putem să începem cu o astfel de organizare însă
dacă vrem să avem un sistem de succes, eventual va trebui să avem o altă abordare.

Într-o arhitectură distribuită formată din mai multe programe, probabil primul gând ar fi să
avem o sursă pentru fiecare microserviciu, numit și multirepo este un mod de organizare
în care fiecare dintre acesta are spațiul și pipeline-ul propriu, astfel se produce 
o separare foarte ușoară între organizare și se cunosc echipele ce lucrează la fiecare
microserviciu. Însă simplitatea acestei separări aduce probleme în alte părți. Reutilizarea
codului devine mai complicată, întrucât fiecare microserviciu va depinde de librării
specializate, dacă microserviciul de la care pornește această librărie face o extensie 
atunci pentru a lansa caracteristicile noi trebuie să așteptăm ca microserviciile ce implementează
librărie să aducă o versiune actualizată. Arhitecturile distribuite contribuie la un tot unitar,
deci uneori o modificare trebuie să fie făcută în cadrul mai multor microservicii, în acest caz
trebuie să avem acces la ambele repo-uri, să facem commit-urile în ambele și după să actualizăm
și UI-ul ce consumă aceste microservicii. Dificultatea modificărilor crește mai ales dacă
e nevoie să fie implicați mai mulți oameni, însă dacă avem astfel de probleme e posibil 
ca modelul de microservicii să fie prea cuplat. Folosirea unei astfel de organizări 
merge suficient de bine indiferent de nivelul la care ne aflăm însă dacă facem modificări
des în mai multe locuri probabil ar trebui să facem câteva modificări la modul de desfășurare.

Un alt mod de organizare este de avea tot codul sursă într-un singur loc. Acesta sună asemănător
cu prima variantă prezentată însă aceasta este configurată și se cunoaște faptul că în cadrul
ei sunt mai multe proiecte. În primul rând, serverele de integrare continuă ar începe să diferențeze
între proiecte în funcție de poziția lor în director, astfel nu ar mai exista probleme cu
build-urile de lungă durată întrucât s-ar face per proiect modificat. Un avantaj al acestui
mod este faptul că reutilizarea codului este mult mai ușoară întrucât fișierele s-ar afla 
într-un loc foarte apropiat, singura dificultate ar fi să includem aceste fișiere în
serverele ce crează build-urile. Totuși separarea între echipe nu este la fel de evidentă,
pe GitHub există un tip special de fișier numit CODEOWNERS în care putem să specificăm 
pentru fiecare folder cine deține elementele, acest lucru ajută întrucât cei menționați
vor fi adăugați automat la recenzie atunci când se face pull request. Acest tip de pattern
are dezavantajul că ar consuma extrem de multă memorie, în același sens ar fi destul de greu
de descărcat mai ales când nu avem nevoie. În acest scop companiile mari își creează unelte
proprii pentru a rezolva aceste probleme, precum Google ce folosește unelte de versionare
a codului propriu, Piper, astfel acest tip arhitectural funcționează foarte bine pentru companiile
mari ce pot investi în îmbunătățirea acestuia dar și pentru echipe foarte mici în care 
separarea între proiecte este ușor de făcut.
