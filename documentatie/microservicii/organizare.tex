\subsection{Consum}

Atunci când construim un sistem, ne așteptăm să îl dăm în folosință pentru cineva,
fie clienți sau ca asistent al unui sistem existent. Însă în general acești clienți pot să
nu consume chiar sistemul direct întrucât acesta poate fi complex, ci putem să îi oferim printr-o
interfață construită tot de noi (în cazul aplicațiilor, putem considera front-end-ul o astfel
de interfață), astfel putem avea ca responsabilitate și crearea unui astfel de lucru.

În modul tradițional de lucru, am avea dezvoltatori grupați în funcție de rolul și platforma pe care
lucrează, astfel toți cei responsabili de o anumită componentă mai mare (de exemplu domeniu
sau prezentare) ar fi în același loc. Acest tip de organizare ne permite să avem o 
experiență identică la același nivel, însă comunicările între nivele pot fi afectate, astfel 
s-ar putea introduce probleme de sincornizare.

O metodă mai eficientă de organizare este gruparea dezvoltatorilor în funcție de locul în care
aceștia lucrează, astfel toți cei responsabili la toate nivelele de o anumită componentă a
sistemului nostru vor lucra împreună. Acest mod ne permite un control mult mai bun, având
metode de schimbări la orice nivel, astfel atunci când avem o inițiativă, aceasta e mai ușor 
de implementat și nu depindem de echipe exterioare pentru care trebuie șă așteptăm 
o implementare.

