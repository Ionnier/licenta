\section{Scopul aplicației}

Chiar dacă o aplicație de tipul acesta nu ilustrează capabilitățile unui sistem
bazat pe microservicii, în urma documentației putem spune că din punct de
vedere arhitectural, folosirea unei acestui tip nu ar aduce
aproape niciun avantaj întrucât există un singur dezvoltator deci segmentarea
ar duce complexitate inutilă iar populația
care va folosi aplicația va fi aproape zero, astfel nu avem nevoie de scalarea
independentă a unor componente. În același timp, deoarece suntem constrânși de resurse
cea mai bună metodă este să delegăm cât mai multă administrare către cloud,
astfel să putem să ne concentrăm cât mai mult pe dezvoltare.

Cu toate acestea, aplicația pe care vreau să o dezvolt are ca scop rezolvarea
unei probleme personale, crearea unei aplicații de administrat task-uri cu diferite
capabilități precum jurnalizarea. În mod mai simplu, vreau să fac o clonă
de Atlassian Jira însă mai simplistă, dedicată oamenilor obișnuiți ce nu au nevoie
de complexitatea unui program de administrare de proiecte.

Ceea ce m-a atras la această idee reprezintă ideea de „Log work”, prima oară
când am interacționat cu această mecanică mi s-a părut interesantă întrucât
se crează un istoric al tuturor activităților pe care le fac, putând
să îmi creez o imagine atunci când mă uit în retrospectivă și să planific mai
ușor în viitor. Alături de aceasta, faptul că am task-uri împărțite sub diferite forme mă ajută
să îmi organizez ziua mult mai eficient.

Jurnalizarea este individuală iar modul în care fiecare vrea să își organizeze
task-urile reprezintă o preferință personală, din acest motiv este dificil să
găsești o aplicație care să se potrivească, din acest motiv nu prea am folosit aplicații
deja create întrucât mi s-a părut că sunt ușor influențat de modul cum ar trebui să
îmi administrez cerințele.

Prin această aplicație vreau să îmi organizez ziua mai ușor, să creez o soluție
de aministrare a lucrurilor care vreau să le fac zilnic ce poate să le aranjeze
automat într-un mod inteligent pentru a-mi oferi o perspectivă asupra zilei.

\section{Aplicatii asemanatoare}

\subsection{Jira}

Jira este una dintre aplicațiile pe care trebuie să le folosesc zilnic
la muncă și din experiența proprie, nu mi se pare că ar îmbunătății colaborarea,
însă personal mă ajută să îmi dau seama de task-urile mai importante, să adaug
notițe pe acestea prin comentarii și să înregistrez numărul de ore cât am lucrat la ele.

Elementele de colaborare sunt de asemenea destul de bune, pot să văd cu ușurință
la ce lucrează alte persoane. Însă Jira încearcă să rezolve problemele administrării
de proiecte, astfel are multe funcții dedicate pentru asta, de exemplu Sprint-uri sau
board-urile Kanban. Din acest motiv, singurul lucru care iesă în evidență atunci
când mă gândesc la Jira sunt sistemul de log work și dashboard-urile customizabile.

\subsection{Google Keep}

O aplicație de mementouri și notițe sub diferite forme. Aceasta este foarte simplistă,
la fel cum sunt majoritatea aplicațiilor de acest tip. Un lucru important la aceasta
este faptul că este accesibilă pe majoritatea dispozitivelor întrucât se prezintă
ca aplicație web dar și nativă de mobil, iar atunci când vorbim de o aplicație de notițe
am vrea să putem să o accesăm oriunde.

Aceasta se concentrează strict pe notițe simple, din acest motiv este limitată. În interiorul
notițelor este dificil să navigăm iar gruparea notițelor se poate face prin adăugarea
unor etichete simple, ceea ce funcționează însă nu este ideal într-o aplicație complexă,
iar posibilitatea de a împărții un task în mai multe sub-task-uri din Jira este lipsită.

\subsection{Todoist}

O aplicație aparent simplă de organizare însă care încearcă să se plaseze ca fiind folositoare
și pentru echipe. Din acest motiv, task-urile pot fi împărțite pe proiecte și ulterior partajate.
Aplicația fie are un design de complex sau nu am petrecut suficient de mult timp cu ea,
însă este dificil de navigat în interiorul ei, iar pentru o aplicație de acest tip este
esențial un design minimalist.

Aceasta are mai multe funcționalități ce o apropie de Jira, precum posibilitatea de a avea
subtask-uri la task-uri, asignarea de priorități, remindere, gruparea pe proiecte
și etichete, existența unui jurnal de activitate ce descrie activitățile pe care le-am făcut
în cadrul aplicației și posiblitatea de distribuire. Prezintă un calendar în interior,
ceea ce e esențial pentru a vedea modul de distribuție a task-urilor.

\section{Cerințe}

Proiectele software urmează o analiză detaliată a specificațiilor pentru a ne
asigura că soluția pe care o implementăm rezolvă problema, însă prefer ca aceasta
să fie cât mai simplă.

Astfel, aplicația trebuie să poată să îndeplinească următoarele cerințe:

\begin{itemize}
    \item Ca și utilizator pot să am posibilititatea de a crea un task pe care vreau să îl completez
    \item Ca și utilizator pot să adaug un subtask la un task creat
    \item Ca și utilizator am posibilitatea de a-mi organiza task-urile sub alte obiective
          generale (asemănător Epic-urilor din Jira)
    \item Ca și utilizator pot să adaug timp petrecut asupra unui task în orice moment
    \item Aplicația va fi minimalistă și adaptabilă, astfel utilizatorul poate să
          își creeze un mod propriu în care aceasta să fie
    \item Aplicația va fi disponibilă pe mai multe dispozitive și va permite sincronizarea
          între dispozitive
    \item Aplicația permite să încardrăm task-urile într-un anumit interval și să vedem asta pe calendar
    \item Aplicația va funcționa integral fără conexiune la Internet și implicit fără niciun server
    \item Aplicația va avea o funcționalitate de planificare automată a zilei inteligentă, în funcție de tipul task-ului însă
          care nu e necesar să respecte condiția legată de conexiunea de Internet
\end{itemize}
