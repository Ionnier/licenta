\section{Scopul aplicației}

Chiar dacă o aplicație de tipul acesta nu ilustrează capabilitățile unui sistem
bazat pe microservicii, în urma documentației putem spune că din punct de
vedere arhitectural, folosirea unei acestui tip nu ar aduce
aproape niciun avantaj întrucât există un singur dezvoltator deci segmentarea
ar duce complexitate inutilă iar populația
care va folosi aplicația va fi aproape zero, astfel nu avem nevoie de scalarea
independentă a unor componente. În același timp, deoarece suntem constrânși de resurse
cea mai bună metodă este să delegăm cât mai multă administrare către cloud,
astfel să putem să ne concentrăm cât mai mult pe dezvoltare.

Cu toate acestea, aplicația pe care vreau să o dezvolt are ca scop rezolvarea
unei probleme personale, crearea unei aplicații de administrat task-uri cu diferite
capabilități precum jurnalizarea. În mod mai simplu, vreau să fac o clonă
de Atlassian Jira însă mai simplistă, dedicată oamenilor obișnuiți ce nu au nevoie
de complexitatea unui program de administrare de proiecte.

Ceea ce m-a atras la această idee reprezintă ideea de „Log work”, prima oară
când am interacționat cu această mecanică mi s-a părut interesantă întrucât
se crează un istoric al tuturor activităților pe care le fac, putând
să îmi creez o imagine atunci când mă uit în retrospectivă și să planific mai
ușor în viitor. Alături de aceasta, faptul că am task-uri împărțite sub diferite forme mă ajută
să îmi organizez ziua mult mai eficient.

Jurnalizarea este individuală iar modul în care fiecare vrea să își organizeze
task-urile reprezintă o preferință personală, din acest motiv este dificil să
găsești o aplicație care să se potrivească, din acest motiv nu prea am folosit aplicații
deja create întrucât mi s-a părut că sunt ușor influențat de modul cum ar trebui să
îmi administrez cerințele.

Prin această aplicație vreau să îmi organizez ziua mai ușor, să creez o soluție
de aministrare a lucrurilor care vreau să le fac zilnic ce poate să le aranjeze
automat într-un mod inteligent pentru a-mi oferi o perspectivă asupra zilei.

\section{Aplicatii asemanatoare}

\subsection{Jira}

Jira este una dintre aplicațiile pe care trebuie să le folosesc zilnic
la muncă și din experiența proprie, nu mi se pare că ar îmbunătății colaborarea,
însă personal mă ajută să îmi dau seama de task-urile mai importante, să adaug
notițe pe acestea prin comentarii și să înregistrez numărul de ore cât am lucrat la ele.

Elementele de colaborare sunt de asemenea destul de bune, pot să văd cu ușurință
la ce lucrează alte persoane. Însă Jira încearcă să rezolve problemele administrării
de proiecte, astfel are multe funcții dedicate pentru asta, de exemplu Sprint-uri sau
board-urile Kanban. Din acest motiv, singurul lucru care iesă în evidență atunci
când mă gândesc la Jira sunt sistemul de log work și dashboard-urile customizabile.

\subsection{Google Keep}

O aplicație de mementouri și notițe sub diferite forme. Aceasta este foarte simplistă,
la fel cum sunt majoritatea aplicațiilor de acest tip. Un lucru important la aceasta
este faptul că este accesibilă pe majoritatea dispozitivelor întrucât se prezintă
ca aplicație web dar și nativă de mobil, iar atunci când vorbim de o aplicație de notițe
am vrea să putem să o accesăm oriunde.

Aceasta se concentrează strict pe notițe simple, din acest motiv este limitată. În interiorul
notițelor este dificil să navigăm iar gruparea notițelor se poate face prin adăugarea
unor etichete simple, ceea ce funcționează însă nu este ideal într-o aplicație complexă,
iar posibilitatea de a împărții un task în mai multe sub-task-uri din Jira este lipsită.

\subsection{Todoist}

O aplicație aparent simplă de organizare însă care încearcă să se plaseze ca fiind folositoare
și pentru echipe. Din acest motiv, task-urile pot fi împărțite pe proiecte și ulterior partajate.
Aplicația fie are un design de complex sau nu am petrecut suficient de mult timp cu ea,
însă este dificil de navigat în interiorul ei, iar pentru o aplicație de acest tip este
esențial un design minimalist.

Aceasta are mai multe funcționalități ce o apropie de Jira, precum posibilitatea de a avea
subtask-uri la task-uri, asignarea de priorități, remindere, gruparea pe proiecte
și etichete, existența unui jurnal de activitate ce descrie activitățile pe care le-am făcut
în cadrul aplicației și posiblitatea de distribuire. Prezintă un calendar în interior,
ceea ce e esențial pentru a vedea modul de distribuție a task-urilor.

\section{Cerințe}

Proiectele software urmează o analiză detaliată a specificațiilor pentru a ne
asigura că soluția pe care o implementăm rezolvă problema, însă prefer ca aceasta
să fie cât mai simplă.

Astfel, aplicația trebuie să poată să îndeplinească următoarele cerințe:

\begin{itemize}
\item Aplicația va fi disponibilă pe mai multe dispozitive și va permite sincronizarea
între dispozitive
\item Aplicația va funcționa integral fără conexiune la Internet și implicit fără niciun server
\item Aplicația va avea o componentă creată cu o arhitectură bazată pe microservicii
    \item Ca și utilizator pot să am posibilititatea de a crea un task pe care vreau să îl completez
\item Ca și utilizator pot să adaug un subtask la un task creat
    \item Ca și utilizator pot să adaug timp petrecut asupra unui task în orice moment
    \item Aplicația va fi minimalistă și adaptabilă, astfel utilizatorul poate să
își creeze un mod propriu în care aceasta să fie
\item Aplicația permite să creăm o planificare a zilei curente realizat din task-urile pe care le-am introdus
    \item Aplicația va avea o funcționalitate de planificare automată a zilei inteligentă, în funcție de tipul task-ului însă
care nu e necesar să respecte condiția legată de conexiunea de Internet
\end{itemize}

\section{Arhitectura}

Centrul aplicației va fi cluster-ul de Kubernetes. Pentru a ușura dezvoltarea vom folosii
un cluster administrat de un provider de cloud, în cazul meu Digital Ocean Managed Kubernetes.
Astfel pot să administrez un cluster cu ușurință și să modific numărul de noduri în orice moment
fără a fi nevoit să fac vreo modificare manuală în afara centrului de control.

Pentru a lansa microserviciile în Kubernetes va trebui ca acestea să fie stocate
în interiorul unui registru de containere, astfel o să creez unul tot pe Digital Ocean, având
beneficiul că nu va trebui să adminstrez modul în care administrez cheile de acces către acestea,
cluster-ul primind implicit acces către el prin diferite funcții oferite de provider.

Întrucât Kubernetes nu excelează în hostarea de baze de date voi încerca să deleg cât mai multe
funcționalități ce pot fi hostate în exterior cu putință. Astfel, voi folosii o bază de date MongoDB hostată
pe Mongo Atlas și pentru metrici și observabilitate voi folosii serviciile oferite de Grafana Cloud.

În interiorul clusetului de Kubernetes voi rula toate microserviciile prin facilitățile pe care
acesta ni le oferă alături de alte obiecte de care am avea nevoie, precum un message broker sau colector de log-uri.

Aplicația va fi dezvoltată în Flutter pentru a avea un singur code base ce duce la
crearea aplicației pe mai multe dispozitive.

\section{Implementare}

\subsection{Interfață}

În momentul în care am pornit implementarea, am crezut că cea mai mare dificultate
pe care o să o am în proiect va fi administrarea clusterului de Kubernetes, provenit din experiențele
precedente, astfel am decis că va fi o idee bună în a începe cu interfața.

Interfața principală va fi aplicația creată în Flutter. O dificultate pe care am avut-o
a fost faptul că deși eu știam în mare măsură ce vreau ca aplicația să facă, eu nu am definit
modul cum aș vrea ca aceasta să arate. Întrucât pe parcursul programului de licență nu am avut
proiecte care să se axeze pe aspectul și design-ul unei aplicații, am ajuns la un obstacol
în momentul în care am început implementarea. În mod inițial, scopul aplicației a fost să
creez ceva ce aș putea să folosesc în continuare (de aici vine cerința de a nu avea
dependințe pe niște servere, întrucât nu aș putea să le hostez mereu), însă nefiind satisfăcut
de modul cum afișez interfața a trebuit să renunț la obiectivul de a face ceva util, însă totuși
va putea fi utilizată într-o anumită măsură, deși nu ar exista neapărat un motiv pentru asta.

Va exista și o interfață ce are cu rol în autentificare și administrare a contului. O abordare generală
pe care o putem observa în aplicațiile pe care le folosim este că ele apelează același serviciu
prin care se autentifică, de exemplu când vrem să ne conectăm la un produs Google, mereu
vom fi redirecționați către accounts.google.com unde vom face autentificarea. În același mod,
atunci când vrem să facem o modificare asupra unui cont (de exemplu schimat de parolă sau a unui mail)
nu o vom facem din aplicația propriu-zisă ci tot de pe acel site de management.

Am vrut să replic acest comportament, deci va exista o interfață web minimalistă, implementată în Svelte
ce va rula în interiorul clusterului prin care vom interactiona cu microserviciul de autentificare.

\subsection{Autentificare}

Autentificarea în aplicație se va face prin intermediul microserviciului de autentificare. Acesta este
conectat la o bază de date MongoDB hostată pe Mongo Atlas unde va reține datele de care are nevoie.

Vom putea face autentificarea prin intermediul interfeței web, care atunci când se conectează cu succes
va înapoia un JSON Web Token ce conține datele necesare pentru celelalte microservicii, iar atunci când utilizatorul
încearcă să intre pe o cale protejată va trebui să prezinte acest token.

\subsection{Sincronizarea}

Pentru a îndeplinii cerința de sincronizare pe mai multe dispozitive, va trebui creat un microserviciu
ce are ca scop memorarea datelor utilizatorului. În crearea acestei funcționalități
am întâmpinat dificultăți în crearea unei sincronizări adevărate întrucât logica
era deja implementată în interfață atunci când am ajuns la acest pas.

Astfel, pe dispozitivele mobile, datele sunt menținute într-un fișier prin intermediul SQLite3, astfel,
serviciul de sincronizare reprezintă doar o modalitate de a urca acest fișier cu date. Acest lucru
nu va duce la performanțe bune, dar va ușura dezvoltarea.

\subsection{Timeline}

Nu reprezenta o cerință, însă pentru a ilustra comunicarea prin evenimente am creat un Timeline, asemănător cu cel
de pe Jira, în care atunci când cineva își actualizează un task, va fi vizibil pentru toți într-un Activity Log.

Am creat un microserviciu care va memora persoanele care sunt prietene cu celelalte persoane (sau un sistem de urmărie).
În cadrul acestuia, se vor memora ID-urile persoanelor comune. La un anumit interval acest microserviciu își va actualiza datelele,
astfel va face o cerere către microserviciul de sincronizare pentru a salva intern anumite date despre task-urile la care s-a lucrat.

\subsection{Profile}

Pentru a complementa microserviciul de Timeline, am creat un microserviciu ce generază profilul unor persoane.
Pentru a genera profilul cuiva, am avea nevoie de informații ce provin din microserviciul de autentificare pentru
datele acestuia precum nume sau adresa de email dar și din cel de Timeline pentru a vedea toate persoanele
cu care e prieten. Astfel, se fac mai multe cereri decât am avea nevoie.

Pentru a rezolva această problemă, putem folosii o arhitectură precum RestQL unde
putem agrega cererile si să specificăm ce informații să primim.

Implementarea mea ilustrează arhitectura Backend For Frontend în care am construit un microserviciu
care face cererile necesare și împachetează rezultatele pentru a efectua cererile.

\subsection{Comunicarea intre microservicii}

Microserviciile vor fi lansate în cadrul Kubernetes, astfel acestea vor comunica 1 la 1 prin intermediul
DNS-ului ce funcționează în interiorul cluster-ului pe care Kubernetes îl administrează.

Va exista un message broker Rabbit MQ ce are ca scop distribuirea evenimentelor precum atunci când
un utilizator schimbă informații despre contul acestuia sau atunci când își sincronizează baza de date.

În acest mod, celelalte microservicii își pot actualiza datele pe care le memorează,
de exemplu cel de timeline va face o cerere din nou pentru a actualiza datele utilizatorului ce își actualizează baza de date.

\subsection{Microserviciul de sugestii}

TODO: //

\section{Automatizări}

Codul va fi versionat prin intermediul GitHub, astfel folosind GitHub Actions vom putea crea
automatizări în funcție de lucrurile ce se întâmplă asupra codului nostru.

Astfel, putem să:

\begin{itemize}
    \item Atunci când publicăm un tag nou, se va crea un build al aplicației de Flutter pentru Android
          (reprezintă cea mai simplu de implementat) ce va fi stocat în GitHub Releases
    \item Atunci când publicăm modificări asupra codului unui microserviciu, iar acesta intră în branch-ul principal,
          se va declanșa un build al imaginii care va fi urcat automat in repository-ul de pe Digital Ocean
\end{itemize}

Ca o completare ale acestor automatizări putem să creăm ca atunci când modificăm
fișierele de Kubernetes, acestea să fie aplicate automat pe cluster, dar și
să implementăm teste care să ruleze în perioada de build.

