\chapter{Introducere}
Domeniul informaticii, calculatoarelor și al tehnologiei informației este unul care va fi mereu
în continuă evoluție. Dorința companiilor de a-și servi clienții cât mai rapid a dus la un avans
tehnologic în care mereu apare ceva nou ce are ca scop îmbunătățirea proceselor actuale. Această arie și-a
început dezvoltarea aproape acum o sută de ani iar avansul poate fi apreciat numai gândidu-ne la modul în
care scriam cod și livram produse acum, acum zece ani și acum două zeci de ani. Se observă o diferență
și la accesibilitatea unităților de calcul, actual majoritatea persoanelor au acces la un dispozitiv ce
se poate conecta la Internet, ceea ce le permite să devină consumatori la diferite servicii.

Piața pentru unități de calcul a evoluat. În 1943, Thomas Watson, director la IBM menționează că 
„I think there is a world market for maybe five computers” (Cred că există o piață globală 
pentru poate cinci computere). În 1999, Michael Barr afirmă în „Programming Embedded Systems in C and C++”
„One of the more surprising developments of the last few decades has been the
ascendance of computers to a position of prevalence in human affairs. Today there are
more computers in our homes and offices than there are people who live and work in
them. Yet many of these computers are not recognized as such by their users.” (Una dintre cele mai 
surprinzătoare dezvoltări a ultimelor decenii a fost ascensiunea computerelor într-o posiție predominantă 
în afacerile oamenilor. Astăzi sunt mai multe computere în casele noastre și în birouri decât persoane care trăiesc
și lucrează în ele. Însă multe dintre acestea nu sunt recunoscute de către utilizatorii lor). Cele două fraze evidențiază
modul cum s-a schimbat prezența computerelor în viața noastră. La început acestea erau foarte costisitoare, 
greu de administrat și de operat însă acestea au devenit din ce în ce mai mici iar recent aproape orice are 
nevoie de semiconductori datorită integrării ce le fac „smart”. 

Aceste afirmații se aplică și pentru produsele software, ce au ca menire să ofere asistență altor aplicații
sau să fie consumate direct. Indiferent de locul în care mă duc, probabil persoana cu care aș interactiona
folosește un computer, fie că îmi cumpăr ceva de la un magazin făcând o plată cu cardul sau că mi se livrează un colet
iar curierul marchează pe AWB că a fost livrat. Christopher Little afirmă că „Every company is a technology company, regardless of 
what business they think they are in. A bank is just an IT company with a banking license.” (Orice companie
este o companie ce se axează pe tehnologie, indiferent de mediul de afaceri în care se află. O bancă este doar o companie
IT cu licență de a funcționa ca o bancă.)

% evaluarea companiilor bazata pe livare
% faptul ca toate companiile vor sa fie performante fata de competitori