\chapter{Introducere}
Domeniul informaticii, calculatoarelor și al tehnologiei informației este unul care va fi mereu
în continuă evoluție. Dorința companiilor de a-și servi clienții cât mai rapid a dus la un avans
tehnologic în care mereu apare ceva nou ce are ca scop îmbunătățirea proceselor actuale. Această arie și-a
început dezvoltarea aproape acum o sută de ani iar avansul poate fi apreciat numai gândidu-ne la modul în
care scriam cod și livram produse acum, acum zece ani și acum două zeci de ani. Se observă o diferență
și la accesibilitatea unităților de calcul, actual majoritatea persoanelor au acces la un dispozitiv ce
se poate conecta la Internet, ceea ce le permite să devină consumatori la diferite servicii.

Piața pentru unități de calcul a evoluat. În 1943, Thomas Watson, director la IBM menționează că 
„I think there is a world market for maybe five computers” (Cred că există o piață globală 
pentru poate cinci computere). În 1999, Michael Barr afirmă în „Programming Embedded Systems in C and C++”
„One of the more surprising developments of the last few decades has been the
ascendance of computers to a position of prevalence in human affairs. Today there are
more computers in our homes and offices than there are people who live and work in
them. Yet many of these computers are not recognized as such by their users.” (Una dintre cele mai 
surprinzătoare dezvoltări a ultimelor decenii a fost ascensiunea computerelor într-o posiție predominantă 
în afacerile oamenilor. Astăzi sunt mai multe computere în casele noastre și în birouri decât persoane care trăiesc
și lucrează în ele. Însă multe dintre acestea nu sunt recunoscute de către utilizatorii lor). Cele două fraze evidențiază
modul cum s-a schimbat prezența computerelor în viața noastră. La început acestea erau foarte costisitoare, 
greu de administrat și de operat însă acestea au devenit din ce în ce mai mici iar recent aproape orice are 
nevoie de semiconductori datorită integrării ce le fac „smart”. 

Aceste afirmații se aplică și pentru produsele software, ce au ca menire să ofere asistență altor aplicații
sau să fie consumate direct. Indiferent de locul în care mă duc, probabil persoana cu care aș interactiona
folosește un computer, fie că îmi cumpăr ceva de la un magazin făcând o plată cu cardul sau că mi se livrează un colet
iar curierul marchează pe AWB că a fost livrat. Christopher Little afirmă că „Every company is a technology company, regardless of 
what business they think they are in. A bank is just an IT company with a banking license.” (Orice companie
este o companie ce se axează pe tehnologie, indiferent de mediul de afaceri în care se află. O bancă este doar o companie
IT cu licență de a funcționa ca o bancă.)

Desigur nu toate companiile sunt la fel și nu toate concurează cu toate companiile, însă chiar dacă 
nu ești o companie ce activează în tehnologie, probabil folosești componente tehnologice pentru a-ți
îmbunătății randamentul sau să oferi clienților un avantaj față de competiție, care probabil gândește în
același mod.

Însă pentru fiecare companie obiectivul este identic, să livreze clienților produse de calitate și cât mai 
rapid. Pentru aceasta trebuie să ne asigurăm că avem inginerii necesari pentru implementarea cererilor noi,
însă momentul în care aceasta este terminată este doar începutul procesului de integrare și ulterior lansare,
și uneori acesta este cel care reprezintă un blocaj din mai multe puncte de vedere ce limiteaza numarul de câte
ori putem aduce o îmbunătățire produsului nostru.

În încercarea de a livra mai rapid sau pentru a ușura modul în care manevrăm un influx sau o lipsă de trafic,
s-a încercat folosirea unei arhitecturi paralele asupra aplicațiilor creând arhitecturi orientate pe servicii, 
iar în momentul în care aceste servicii sunt concentrare pe un număr redus de funcționalități și pot fi lansate
independent și să își păstreze funcționalitate putem vorba de o arhitectură bazata pe microservicii.

Însă schimbând arhitectura sau practiciile din modul de desfășurare al livrarii produsului nu este suficient
pentru a asigura performanța, întrucât ambele dintre ele aduc dezavantaje ce trebuie tratate iar uneori 
acestea sunt mai mari decât avantajele pe care schimbarea le-ar aduce, însă implementarea corectă în locurile
ce ar beneficia de o astfel de abordare poate să aducă performanțe ce nu ar fi posibile cu procesele vechi.

Afinitatea mea pentru microservicii provine de la lipsa de cunoștiințelor pentru a putea alege dintre mai
multe limbaje de programare, framework-uri sau platforme. De exemplu, ce avantaje mi-ar face să aleg ca 
pentru server-ul meu să folosesc o aplicație în Node.JS sau una in Go? Chiar dacă putem să cunoaștem 
niște lucruri informative la început, cei mai buni indicatori sunt monitorizările proprii în producție. De 
asemenea pot fi cazuri în care îmbunătățirile aduse unei platforme pe parcurs o fac mai bună față de ce era
inițial (dezvoltarea TypeScript poate să influențeze alegerea inițială a unei platforme SpringBoot ce este type-safe).
Astfel, pe o arhitectură bazată pe microservicii putem să folosim limbajul potrivit pentru serviciul căruia
îi aduce cele mai multe beneficii. Acest lucru se extinde și pentru baze de date. Poate în timpul evoluției 
serviciului apar tehnologii noi, însă având un singur lucru central ar face migrarea mult mai grea, de asemenea
uneori ar trebui să facem compromisuri pe care alegerea unui alt tip de baze de date ar face să o dispară.

Dezvoltarea tematiciilor DevOps pornește de la ușurința cu care ne putem crea o bază de date folosind containere,
întrucât instalarea unei baze de date pe calculatorul personal în timp ce lucram la proiectele de facultate a făcut
să am experiențe în care la finalul fiecărui an îmi reinstalam sistemul de operare doar pentru că o dată ce instalam
o bază de date crea suficiente servicii, foldere care dupa o dezinstalare încă rămâneau. Folosind Docker,
pot crea o bază de date cu o singură linie de cod, care în același timp să fie preconfigurată și să o pot
împărtășii cu colegii de proiect ca să nu pierdem timp în configurarea mediului de lucru. 

Lucrarea mea va urma o structură în care aprofundez aspectele teoretice la început iar la final încerc să
aplic elementele prezentate în crearea unei aplicații. Înițial pornesc cu prezentarea microserviciilor
continuând cu diferitele moduri în care acestea pot fi lansate, imediat după cu dezvoltarea noțiuniilor de DevOps
ce au ca scop accelerarea dezvoltării, iar la final o prezentare generală a modului în care îmi construiesc aplicația.
