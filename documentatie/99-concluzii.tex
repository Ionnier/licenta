\chapter{Concluzii}

O arhitectură bazată pe microservicii reprezintă o modernizare a modului de a dezvolta aplicații.
Cuplat cu platformele de infrastructură în cloud ce ne permit crearea de servere
în orice moment într-un timp foarte scurt și diferitele tipuri de orchestrare
putem crea aplicații scalabile și reziliente.

Microserviciile sunt o arhitectură de dezvoltare a aplicațiilor în care aplicația este
împărțită în componente independente ce reprezintă o parte indepententă și este responsabilă pentru o
funcționalitate specifică al procesului de afaceri. Aceste servicii comunică între
ele prin intermediul unor interfețe bine definite și pot fi dezvoltate, implementate și scalate independent.

Prin utilizarea arhitecturii bazate pe microservicii,
aplicațiile devin mai ușor de întreținut și scalat,
deoarece fiecare serviciu poate fi actualizat și scalat independent,
fără a afecta restul sistemului.
De asemenea, permite dezvoltarea agilă și
implementarea continuă, deoarece fiecare serviciu poate
fi testat și implementat separat.

Cu toate acestea, arhitectura bazată pe microservicii
introduce și provocări suplimentare în gestionarea
comunicării între servicii și în monitorizarea și
administrarea infrastructurii distribuite. Amazon Prime Video a observat
o creștere semnificativă de performană atunci când au făcut trecerea de la o arhitectură
distribuită către una monolitică \cite{Kolny_2023}. În cazul acestora, comunicațiile
dintre părțile componente au dus la consum suplimentar de resurse.

Deși nu unica metodă de orchestrare pentru microservicii, Kubernetes reprezintă o metodă complexă de orchestrare a containerelor ce oferă
un mediu scalabil și flexibil pentru implementarea aplicațiilor. Oferă diferite
funcționalități esențiale precum descoperirea în interiorul cluster-ului,
posibilitatea de a gestiona actualizări ale configurației sub formă de fișiere ce permite folosirea
acestuia împreună cu sisteme de gestiune a infrastructurii prin cod dar și
distribuirea sarcinilor în mod inteligent și rutarea traficului în interiorul clusterului.

Serviciile diversificate oferite de distribuitorii de infrastructură în cloud
accelerează dezvoltarea, astfel putem evita implementarea
unor funcționalități comune, crearea infrastructurii sau configurarea bazelor de
date la cerere. Astfel putem să ne concentrăm pe construirea de soluții și nu pe modul
cum acestea vor fi implementate.

