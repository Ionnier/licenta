\subsection{General}
Conceptul de închiriere de echipament nu ar trebui să surprindă pe nimeni. În general, echipamentele
ce ar necesita o investiție semnificativă pentru o folosire extrem de rară ar permite ca împrumutarea
acestora să nu fie o alegere proastă, de exemplu închirierea unei pereche de schiuri în contextul
în care mergem o pentru prima dată sau o dată la o perioadă îndelungată. Compania ce închiriază schiurile
trebuie să se asigure de buna funcționalitatea acestora iar noi plătim prețul pe care l-am negociat.

Însă atunci când vorbim de infrastructură, aceasta este posibil folosită în fiecare moment pe Internet,
rentarea închirierii ar putea provenii strict de la serviciile pe care distribuitorul le-ar oferi, astfel
unul din motivele pentru care ai opta pentru o astfel de opțiune nu este doar pentru componenta fizică
ci și pentru bonusurile care provin cu acestea.

În lumea infrastructurii cloud, serviciile oferite pot fi variate începând de la resurse de calcul sub formă
de platforme de procesare scalabile și actualizabile în orice moment, spațiu de stocare hostat cu servicii suplimentare
precum criptare la rest sau redundanță, rețelistică ușor definită, sisteme de monitorizare și observare până la sisteme 
de Machine Learning deja antrenate alături de alte servicii precum Text-to-Speech sau servicii de traducere în timp real.

Prin implementarea unui serviciu de infrastructură cloud, complexitatea procurării de echipament este abstractizată
prin adăugarea unor resurse noi virtuale, mentenanța la nivel fizic este asigurată de către provider iar în funcție
de serviciul ales chiar mai mult din administrarea acestora, astfel ne putem concentra doar pe dezvoltarea și simplificând
lansarea.

\subsection{Extinderea limitelor}

În comerțul obiectelor cu amănântul, conceptul de cumpărare în masă la un preț mai avantajos sună ușor nedrept
însă furnizorilor să ofere servicii la prețuri foarte accesibile, din acest motiv uneori poate să fie mai 
rentabil să folosim această abordare.

Aceștia pot să creeze ferme de servere pe care să le ofere spre folosire, care deși nu sunt nelimitate, 
ar acoperi suficiente neajunsuri, astfel putem crea aplicații ce beneficiază de această resursă nelimitată în calcul.

În martie 2014, Titanfall a apărut pentru PC, Xbox One și Xbox 360. Consolele noi apărând în 2013, nu este 
neașteptat ca să existe jocuri disponibile pentru ambele, însă în general acestea sunt dezvoltate
ca țintă consolele vechi. În acest caz, dezvoltatorul Respawn Entertainment lucra asupra versiunii de PC și
Xbox One în timp ce un studio extern, Bluepoint, lucra asupra versiunii vechi.

Portarea unui joc ce vizează o platformă mult mai puternică către una mai slabă vine cu dificultățile ei,
mai ales în cazul acestui joc ce este destul de complex (12 jucători iar fiecare dintre ei poate avea 
un titan pe care îl controlează, alături de aceștia existând și entități necontrolabile ce sunt de ambele părți).

În interviul dat pentru Digital Foundry \cite{leadbetter_2014} aceștia menționează dificultățile portării
incluzând reducerea materialelor (asset) și îmbunătățirea modurilor de randare și procesare întrucât 
Xbox 360-ul este mult mai limitat decât Xbox One-ul.

Însă, un lucru special al acestui joc este faptul că este disponibil doar cu o conexiune activă la Internet,
mai mult decât atât, acesta are doar o componentă ce conține mai mulți jucători (fără o poveste), lucru 
ce în absența anumitor cunoștințe ar face să privesc o astfel de abordare ca o lăcomie din partea
distribuitolui.

Chiar și cu o consolă mai puternică, la începutul unei generații dezvoltatorii nu pot să fructifice
toată această putere, astfel aceștia au creat un parteneriat cu cei de la Microsoft ca pentru toate 
versiunile să se folosească servere ce provin din Microsoft Azure pentru a îmbunătății experiența.

Cel mai simplu mod prin îmbunătățirea performanței a fost trecerea la servere dedicate, în general 
jocurile pentru a economisi prețul infrastructurii foloseau o arhitectură de tip peer-to-peer însă 
în cazul Titanfall acesta nu ar fi fost articol, din interviul \cite{leadbetter_2014} am extras:
"The server performance requirements for Titanfall are pretty high, so you can't run peer-to-peer. 
This also means you can't have a many virtual machines per physical server, 
meaning if you've got a popular game, then you're going to need a pretty big investment in 
server hardware.".

În interviul \cite{seppala_2014} se menționează mai multe detalii în acest aspect,
"We have all of this AI and things flying around in the world; that has obviously let us build a 
different game than we would have if we'd have gone with player-hosted", astfel dispozitivele se pot ocupa doar de procesarea informațiilor primite
și să nu se ocupe și de obținerea acestora
"Because Titanfall's advanced AI is handled by the Azure servers, your Xbox's or PC's innards can be used to achieve more detailed graphics 
and the game's silky-smooth frame rate. The Titan bodyguards, 
dropships and legions of AI-controlled combatants are essentially free 
from a processing-power standpoint.".
În același articol se menționează cum serverele disponibile în mai multe regiuni, capacitatea de scalare 
au putut duce proiectul la bun sfârșit.

În articolul precedent \cite{leadbetter_2014} aceste informați despre avantajele cloud-ului se extind,
întrucât pentru a face performanța suficient de bună a trebuit să calculeze anumite elemente direct în cloud.
"Respawn wrote a new PVS system that worked dynamically, likely because that's much better for iteration but it need additional CPU performance we haven't got on 
Xbox 360. We took a different approach and pre-calculated static visibility using a 
big farm of Xbox 360s; these ran multi-hour visibility calculations which reduced the 
amount of stuff the 360 had to draw without requiring runtime computation and allowed us 
to use mostly the same geometry as the Xbox One version.
Same deal for shadows, pre-calculate. Also creating occlusion geometry to reduce 
pixels shaded, tech art to the rescue there getting those built!"

Exemplul precedent acoperă doar resursele de calcul, însă furnizorii de cloud nu se limitează doar la atât,
aceștia oferă servicii mult mai complexe astfel putem să extindem capacitățile unui dispozitiv folosind 
resursele provenite de pe servere la distanță al căror disponibilitate este mai ridicată făță de ce
am putea aduce pe cont propriu.


