Modurile în care un microserviciu poate fi lansat este destul de important întrucât
decide o bună parte din arhitectura inițială a acestuia, performanța sau capacitățile
pot fi eficientizate prin folosirea unei metode de deploy bună. Evoluția modului de a face rost
de infrastructură a pornit de la achiziționarea și creare echipelor special dedicate ce poate
devenii extrem de costisitor și necesitând mai mult timp, iar acum se poate face
adăugând cardul pe platforme specializate, adăugarea unor metode de configurare și
de securitate și putem avea un server într-un timp foarte scurt a cărui mentenanță la nivel
fizic devine responsabilitatea distribuitorului. Bill Baker, un inginer de la Microsoft
afirmă că la început tratam serverele ca și animale de companie, „le numim, și atunci
când devin bolnave, le îngrijești și le aduci înapoi. Acum ne tratăm serverele le tratăm
ca o ciradă. Le numeri iar atunci când se îmbolnăvesc, le împuști.” (You name them and when they get sick,
you nurse them back to health. [Now] servers are [treated] like cattle. You number them and when they
get sick, you shoot them.). Tehnologiile actuale ne permit să nu avem nevoie să încercăm să
reconfigurăm ci dacă putem salva datele de care avem nevoie, atunci este mult mai ușor să
reconstruim sistemul de la 0 și să îl aducem în starea în care avem nevoie de el.

Pentru a accelera crearea unui produs, nu este suficient să accelerăm dezvoltarea, ci găsirea
unei metode de eficientizare a întregului proces. În trecut, lansarea unui produs era una dintre cele
mai mari încetiniri întrucât includea crearea unui tichet și așteptara ca un specialist
să ne contacteze care ar primi programul și l-ar trimite în execuție. Procesul ar fi anevoios
întrucât nu am putea să lansăm atunci când am vrea ci doar atunci când ni se acordă timp, iar
în funcție de personal aceasta poate dura chiar luni, astfel lansările ar trebui să planificate
din timp și să fie solide, să nu existe necesitatatea de EFX-uri întrucât ar complica organizarea.

Cunoașterea metodelor de livrare și alegerea celei potrivite este esențială întrucât în
funcție de obiectivele pe care le avem, buget și alte considerente putem ajunge ca deployment-ul
să fie automat sau controlat la anumite nivele, astfel livrarea ar putea ajunge nici să nu mai fie
un punct dificil.
