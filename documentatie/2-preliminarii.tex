\chapter{Preliminarii}
Cel puțin până la momentul la care am decis ce temă să aleg pentru lucrarea mea de licență niciun curs de bază 
nu a abordat în detaliu tema pe care vreau să o dezvolt, din acest motiv lucrarea mea de licență nu este doar
aprofundarea unui concept și crearea unei aplicații, ci încercarea familiarizării cu obiectele de lucru în același timp.
Desigur, unele aspecte s-ar putea desprinde și în urma practicii în industrie dar nu a fost suficient ca să
pot să consider că ma cunoștințe de bază în acest domeniu și doar încerc să găsesc lucruri cât mai specifice.

De asemenea, microserviciile sunt o arhitectură destul de nouă ce a luat amploare datorită limitărilor din
punctul de vedere al dezvoltării și scalării aplicațiilor monolitice, astfel în opinia mea, este destul
greu să vizualizezi anumite lucruri fără experiență la prima mână cu limitările acestea. Întrucât serviciul
pe care îl dezvolt nu ar necesita neapărat o arhitectură bazată pe microservicii, ar funcția complet normal
și aș reduce foarte mult din complexitate dacă ar fi dezvoltată în mod normal.

La fel și cu practiciile DevOps, acestea se axează mai mult companiilor care au nevoie de timp de penetrare a
pieții foarte rapid, însă eu doar încerc să mă familiarizez cu setarea și folosirea serviciilor pentru uzurile 
mele proprii. Un lucru important ar fi ca ce folosesc să nu mă limiteze, de exemplu infrastructura de care am 
nevoie va provenii din Azure doar pentru că primesc credite gratuite ca și student, însă în absența lor probabil
aș încerca lucruri cu cost minimal sau chiar nici sa nu le fac, și probabil aș avea același randamanet. 