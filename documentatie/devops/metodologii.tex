Eșecurile în proiectele software sunt scumpe, întrucât acestea necesită investiții
semnificative. De aceea optimizarea modului în care acestea sunt create este esențial,
din acest motiv s-a ajuns la coneperea și adaptarea a diferite metodologii
și framework-uri ce au ca scop oferirea unei baze de inspirație
a unui mod de organizare ce poate fi considerat că ar crește șansele
ca proiectul să fie un succes.

\subsection{Waterfall}

Orice obiect destinat consumului, de exemplu diferite unelte sau alimente,
au un proces standard de dezvoltare care cuprinde într-un mod foarte simplist
o linie de ansamblare în care se introduc materialele necesare producerii
iar multe persoane contribuie pentru ca în final să se obțină un singur produs final.

Aplicațiile software sunt diferite întrucât acestea funcționează mai asemănător
cu un serviciu, acesta este creat o singură dată iar apoi este distribuit
către consumatori. Astfel, munca efectivă este creată o singură dată pentru
ansamblarea acestuia.

Însă de-a lungul timpului, modalitatea de distribuție a fost cea care a evoluat,
dacă la început livrarea produselor era foarte asemănătoare cu livrarea
produselor obișnuite, în care exista un produs declarat final și care
era livrat sub formă imutabilă, de exemplu pe CD-uri, acum livrarea se
face doar încărcând produsul într-un anumit loc, sau lansând direct
într-o platformă cloud.

Observăm faptul că o dată ce lansăm produsul, modificarea unor eventuale
probleme este complexă, necesită retrimiterea prin formatul inițial și eventual
chemarea înapoi a lucrurilor deja livrate, acest proces este complex și costisitor.

Din acest motiv este necesar ca produsul dorit să fie descris clar
pentru a elimina eventualele modificări ce ar putea apărea și trebuie să fie
bine testat întrucât corectarea unor greșeli dupa livrare devine imposibilă,
astfel metodologia Waterfall este liniară, are un stagiu de start și unul
de final și se axează pe documentație ce descrie produsul și pe validarea acestora.

Etapele metodologiei Waterfall:

\begin{enumerate}
    \item Crearea specificațiilor, include comunicarea între client și echipă
          a condițiilor ce proiectul ar trebui să le îndeplinească. Acestea trebuie să
          fie foarte clare și să nu lase aspecte ce pot fi interpretate întrucât
          acestea teoretic nu pot fi schimbate în perioada dezvoltării
    \item Crearea arhitecturii, include interpretarea cerințelor clientului
          în vederea creării design-ului proiectului și alcătuirea specificațiilor tehnice ale
          produsului
    \item Implementare
    \item Validare
    \item Mentenanță, după validare proiectul este trimis către client iar acesta
          este considerat final, însă clientul poate revenii pentru a aduce îmbunătățiri
          produsului.
\end{enumerate}

\subsection{Agile}

Metodologia Waterfall este rigidă și nu permite modificări în timpul dezvoltării,
însă acest aspect nu este compatibil cu dezvoltarea software întrucât cerințele
clienților în general nu sunt clare sau se schimbă. În același timp unele
caracteristici trebuie să fie livrate la un anumit timp pentru a avea efectul dorit,
astfel dacă perioada este depășită nu pot să aibă același impact.

Însă în același timp, nevoia de livrare a unui produs complet a scăzut,
întrucât costul de livrare nu mai este imutabil a scăzut necesitatea
verificărilor intense și alcătuirea documentației din Waterfall întrucât
acestea se pot schimba.

Agile se bazează pe crearea unui Minimum Viable Product (MVP) cât mai rapid,
astfel clientul poate să aprecieze modul cum arată produsului.

Pentru a capta aceste schimbări, dezvoltarea în agile se face prin intermediul
unui set de principii și valori ce se axează pe crearea unui mediu în care schimbările
sunt acceptate în orice moment iar părerea clientului este importantă, de aceea
livrarea către acesta trebuie să fie cât mai desă iar produsul trebuie să fie
mereu într-o stare în care poate să fie livrat.

Agile scrum este un framework ce aduce adăugări precum ședințele de planificare,
retrospectivă, ședințe zilnicie în care echipa comunică la ce lucrează și
eventualele probleme pe care le au dar și diferite roluri din cadrul echipelor.

\subsection{DevOps}

Deși nu ar trebui să fie încadrată ca o metodologie, DevOps este în primul
rând o mișcare culturală ce încearcă să rezolve unele probleme de natură
socială ce apar în dezvoltarea aplicațiilor software.

Poate fi considerată un super-set al Agile întrucât majoritatea principiilor
declarate se pot aplica și atunci când încercăm să ne orientăm spre adaptarea
la o astfel de acțiune.

Întrucât nu există o definiție clară, poate fi deficil de estimat dacă
adoptăm DevOps corect, în general putem considera că dacă folosem anumite
practici precum CI/CD sau încercarea optimizării modurilor de lansare am putea
să considerăm că este suficient, însă este incomplet întrucât DevOps
promovează mai mult cultul organizațional și armonizarea relațiilor dintre
persoane care duc la îmbunătățirea perfomanțelor.
