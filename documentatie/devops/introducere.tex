Cea mai populară dintre utilizăriile cunoștințelor acumulate prin învățarea 
programării probabil este dezvoltarea de aplicații software în diferite forme, 
fie în cadrul unei interprinderi sau individual. Pentru a avea orice fel de succes,
aceasta trebuie să aducă valoare unui public iar această necesitate este constrânsă
de diferite aspecte manageriale și de marketing, exprimând în mod simplu, trebuie
ca produsul să răspundă la nevoile clienților în timp util.

Crearea de produse software pentru consumatori sau interprinderi, este o practică ce 
poată să fie considerată suficient de nouă, atribuită perioadei în care computerele 
au putut să fie accesibile și pentru aceștia iar ei vedea valoare în achiziționare 
a acestuia.

Astfel, la fel ca alte aspecte din industria IT, multe concepte au fost preluate
din practici deja existente, chiar dacă cumpărarea de aplicații software sau utilizarea
acestuia este recent, oamenii au fost nevoiți să negocieze alimente sau obiecte
pentru diferite întrebuințări de foarte mult timp. Însă datorită schimbărilor
ce au dus la dezvoltare tehnologică dar și a formei în care aceste aplicații sunt consumate
s-a observat o constantă nevoie de îmbunătățire ale livrării pentru a 
răspunde la nevoile clienților în timp util.

Însă concentrându-ne doar pe produs și pe modalitatea creării acestuia poate să
nu fie benefic. Produsele software au început să devină complexe, nu mai sunt
de independente și au diferite dependințe, ceea ce determină necesitatea apariției
comunicării între echipe diferite ce se ocupă de părți diferite ce susțin
buna funcționare a proiectului.

Din acest motiv, de-a lungul timpului au apărut diferite tendințe dar și idei 
ce au ca principal obiectiv îmbunătățirea modului în care interacționăm cu procesele
și componentele ce duc la formarea produsului final. 