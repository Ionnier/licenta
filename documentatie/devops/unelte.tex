Un aspect pe care se concentrează majoritatea persoanelor atunci când vorbesc
despre DevOps reprezintă uneltele sau aspectele practice pe care urmarea
acestei mișcări o susține, însă nu este suficient mai ales că folosirea
unor unelte nu garantează o creștere a eficienței.

Însă un lucru este cert, pentru a crește eficiența din cadrul unei echipe este
necesar să automatizăm cât mai multe aspecte, de aceea microserviciile nu pot fi
folosite eficient fără a avea o bază formată ce automatizează aspecte precum
testarea și lansările.

Pornind de la nivelul cel mai jos, folosirea unei metode de versionare a codului
reprezintă un standard, aceasta se face prin intermediul Git, iar hostarea repository-urilor
poate să fie făcută pe o platformă precum GitHub, GitLab sau BitBucket. Toate acestea
ar trebui să fie accesibile de orice persoană ce are ca responsabilitate un produs.
Aceasta ne poate ajuta să cunoaștem alte produse și să aducem modificări, chiar și din
exterior atunci când e nevoie prin intermediul pull request-urilor.

Pentru a elimina probleme ce apar datorită compilării pe un dispozitiv local,
putem folosii Maven pentru a automatiza procesul de build, astfel putem să fim siguri
că programul este lipsit de dependințe locale.

Stocarea și versionarea produselor rezultate după build, poate fi făcută
cu un program asemănător cu Nexus Artifact Management, acesta poate fi folosit
pentru fișiere jar, war sau pachete npm și oferă suport pentru controlarea
accesului. Acesta ne permite să reutilizăm rezultatele pasului de build în mai multe locuri.

Un alt aspect pe care putem să îl automatizăm conturează integrarea continuă, o astfel
de unealtă reprezintă Jenkins ce este folosit pentru a sincroniza branch-urile,
rularea de teste automate asupra lor și eventual construirea artefactului și trimiterea
acestuia spre stocare.

Automatizarea infrastructurii și menținerea acesteia sub formă de cod reprezintă
un alt proces ce poate să fie automatizat. Menținerea sub această formă ne aduce
avantaje precum notificarea tuturor personalor implicate prin intermediul pull request-urilor
dar și crearea unui istoric ce poate fi folosit în cazul unui audit. Pentru a automatiza
modul în care procurăm infrastructură putem folosii Terraform, o unealtă folosită
pentru a crea resurse în cloud prin intermediul codului oferind starea pe
care vrem să o avem. Acesta poate fi folosit pentru majoritatea platformelor cloud dar și
peste echipament on-premise.

O dată ce echipamentul este procurat, o altă dificultate reprezintă configurarea acestora. În trecut
acesta reprezenta un lucru manual extrem de repetitiv, însă atunci când avem de a face
cu un astfel de proces pot exista probleme ce apar din erori umane, astfel
automatizarea ar ajuta prin evitarea acestora. O altă unealtă pe care putem să o folosim
în acest scop reprezintă Ansible, ce ne ajută în configurarea automată a infrastructurii
prin intermediul codului.

Atunci când trebuie să folosim părți de aplicație ce nu au un artefcat clar (de
exemplu un server în Node.JS) putem folosi containere Docker pentru a le stoca
sub forma unui container, astfel acesta poate să fie portat dintr-un mediu în altul și
să fie refolosit la mai multe stagii.

Pentru a orchestra modul în care containerele rulează putem folosii Kubernetes, acesta
ne permite să administrăm cluster-ul și avem facilitatăți precum DNS în interiorul acestuia,
posibilități de recuperare și orchestrare a lansărilor.

Monitorizarea reprezintă un alt aspect important pentru a estima starea sistemului,
de aceea putem folosi ceva asemnănător cu stack-ul ELK. Elasticsearch reprezintă
o bază de date cu posibilitate de căutare pentru a stoca elementele de jurnalizare.
Logstash oferă o metodă de procesare a elementelor de jurnalizare pe care le
preia din diferite surse și pe care le trimite în Elastic. Kibana reprezintă o platformă de vizualizare a
datelor stocate în Elastic.

Pentru a automatiza testarea putem folosii framework-uri precum Selenium, astfel
aceste teste vor rula de fiecare dată când se creează un build. În același stil
putem folosi unelte de analizare a seucrității prin integrarea acestora în interiorul
ciclului.

Testarea se face la diferite nivele, astfel testarea automată poate să aducă
un sens de securitate pentru integrarea schimbărilor, acestea fiind vizibile
pe pull request. În același stil, pentru testarea manuală putem să creăm medii efemere
ce sunt create strict pentru a testa acel branch. Pentru a face asta putem folosii Vagrant.

Atunci când avem multe sisteme, va exista necesitatea administrării certificatelor,
parolelor și cheilor de API externe, acestea trebuie să aibă o anumită perioadă de timp
cât sunt valabile iar în cazul în care sunt compromise trebuie să aibă posibilitatea
să fie schimbate și să se reflecte acestu lucru în sistemele deja existente. Pentru a rezolva
această problemă putem folosii HashiCorp Vault ce oferă aceste funcționalități.

Chiar dacă sunt mai puțin vizibile, sistemele de comunicare sunt altă parte importantă a
DevOps-ului, fie că acestea reprezintă un sistem de Mail sau o platformă de
comunicare în timp real precum Slack, acestea trebuie să îmbunătățească posibilitatea
de sincornizare a muncii angajaților dar și de crearea unei căi de alertare în cazul
apariției erorilor. În același timp, acestea trebuie să fie folosite pentru a păstra
la curent persoanele care nu pot participa la ședințe sau lucrează de la alte sedii sau de
acasă.

O bună parte din automatizări se referă la infrastructură și la configurarea
acesteia, din acest motiv este util dacă folosim servicii administrate
de către providerii cloud prin platformele acestora. Acestea oferă multe alte
servicii pe lângă sistemele clasice și pot oferii servicii de notificare și monitorizare complexe
ce ușurează ciclul de dezvoltare.
