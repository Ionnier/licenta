Chiar și în timp ce citeam despre DevOps toate elementele din jurul acestuia
deveneau din ce în ce mai neclare. Aceasta poate să fie și din faptul că
discută despre elemente pe care eu nu am putut să le experimentez personal, mai ales
cele legate de organizație sau de unelte al căror întrebuințare depășesc nevoile
proiectelor asupra cărora vreau să lucrez.

Din acest motiv, poate să fie ușor neclar ce reprezintă DevOps, și asta poate să
fie văzut și în resursele disponibile, întrucât nu există o anumită formalitate
asupra acestui termen.

Un lucru este cert, DevOps este o mișcare ce dorește să mobilizeze personalul
de a îmbunătății modul cum interacționează cu procesul de dezvoltare prin diferite forme,
îmbunătățirea comunicării, creșterea nivelului de automatizare și delegarea responsabilităților
către diferite servicii.

Astfel, atunci când vrem să adoptăm acest concept este suficient doar să îmbunătățim treptat
lucrurile la care lucrăm și care ne influențează, ignorând mentalități precum
păstrarea modurilor antice dar care funcționează pentru a crește randamentul procesului.

