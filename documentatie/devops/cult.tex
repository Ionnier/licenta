Primul punct pe care DevOps se concentreaza sunt aspectele culturale ale companiei.
Însă acest lucru nu este nou, management-ul poate fi considerată o ștință
ce se ocupă cu descoperirea aspectelor desfășurate într-o companie ce o cauzează
să aibă succes, iar o mare parte din acestea o reprezintă relațiile pe care
angajații o au la locul de muncă.

DevOps a fost elaborat pentru a sporii randamentul pe care echipele o au atunci
când interacționează cu produsul. Însă orice aplicație probabil se va conecta
la o bază de date prin intermediul unui serviciu web, tipica arhitectură pe trei straturi.
Din acest motiv, multe echipe cu roluri diferite trebuie să interacționeze,
dezvoltatorii, testerii, oamenii ce se ocupă de lansări (operatori), persoanele
ce administrează infrastructură, alături de aceștia sunt diferite persoane din
lumea de afaceri, manageri, deținătorii de produs, agenții de vânzării, echipele de suport
și alti stakeholders.

Atunci când avem multe echipe independente ce lucrează pe un singur produs
poate devenii dificil să lansăm caracteristici noi prin metode clasice
dacă ignorăm modul în care tratăm produsul.

Istoria cuvântului DevOps, în mod simplist am putea spune că provine de
la armonizarea dintre Dezvoltatori și Operatori. Acest grup de persoane
pot fi considerate că sunt „dușmani” ce sunt forțați să interacționeze
pentru a crea produsul final. Relația această este creată de modul în care
cele două roluri sunt promovate, Dezvoltatorii vor fi mereu celebrați pentru
crerea de caracteristici, rezolvarea de probleme și sporirea potențialui
valori pe care produsul creat ar putea să o aducă, din acest motiv
aceștia vor să lanseze în piață cât mai des. Operatorii sunt promovați
pentru stabilitatea produsului, astfel clienții trebuie să poată să își acceseze
serviciul în orice moment, astfel este pusa foarte mult presiune pe aceștia
ca serverele să fie operabile, iar lansările sunt punctele cele mai critice
întruât orchestrarea acestora poate să fie dificilă și poate să cauzeze
probleme pentru consumatorii.

O astfel de relație apar și între dezvoltatori și echipa de testare,
vizibil în modul Waterfall unde dezvoltatorii trebuie să se întoarcă
să lucreze la un proiect pe care ei îl considerau terminat, sau între
dezvoltatori și echipele de securitate ce testează vulnerabilitățile.
Întrucât acest tip de verificare este făcut abia la final, reînceperea
dezvoltării pentru a fixa anumite probleme poate să fie costisitoare.

Armonizarea unei astfel de relații este cea care această mișcare dorește
să o rezolve, și nu este singura. Studiile de management dar și intuiția
ne pot arăta că mediul în care lucrăm ne ajută să performăm. Dezvoltarea
aplicațiilor reprezintă o metodă de inginerie ce este extrem de taxantă asupra
persoanelor implicate, din acest motiv crearea unui mediu de lucru ideal
este importantă iar aceasta poate pornii de la cultul organizațional.

Însă fixarea unor astfel de probleme necesită ca mediul de lucru să
permită schimbări, fie ca managerii să accepte o perioadă de ineficiență
iar persoanele implicate să fie comunicativi și să accepte reunțarea
la unele obiceiuri vechi ce funcționează însă care pot să fie considerate ineficiente.

O astfel de metodă poate fi împuternicirea echipelor prin creare unor echipe
funcționale cross funcționale ce le permite participanților să facă mai multe
elemente din cadrul produsului, astfel cunoaștem mai multe părți ale acestuia.

Dezvoltarea software pornește de la lucrurile comune pe care le știam, astfel
putem să aplicăm strategiile pe care companiile de dezvoltare le aplică asupra
construiri obiectelor dificle precum mașinile.

O metodă ce a revoluționat dezvoltarea în Japonia, a fost producția de tip lean
apărută la Toyota ce se axează pe maximizarea productivității și minimizarea
pierderilor, adică a tuturor aspectelor ce nu generează valoare
pentru client. Pe scurt, aceasta se bazează pe un proces de
dezvoltare continuă pe baza unui set de principii, identificarea modalităților
de apariție a valori, identificarea circuitului necesar pentru crearea valorii
de la obținerea materiilor prime până la produsul final, crearea unui
flux pentru a minimiza timpii morți și a unui sistem de creare a strictului
necesar pentru a nu consuma timp inutil pe elemente ce nu aduc valoare
și perfectizarea întregului proces în orice moment pentru a rămâne competitivi.

Sunt multe aspecte pe care le putem desprinde din acest mod de dezvoltare
și să îl aplicăm în tehnologie, însă mi se pare că cel mai important dintre
acestea reprezintă crearea acestui circuit al valorii. Atunci când am vorbit despre
această dezvoltare de software am vorbit despre aceste echipe diferite
ce lucrează toate pentru a aduce valoare printr-un singur produs.

Există multe situații în care rezolvarea unor probleme necesită atingerea
unor puncte diferite din cadrul sistemului, astfel o abordare DevOps dorește
să împuternicească persoanele implicate, dându-le posibilitatea de a cunoaște
și alte sisteme pe lângă cele asupra celor care lucrează. De exemplu, fiind
dezvoltator a unei aplicații software ce încearcă să rezolve o problemă
de natură de server, posibilitatea de a vizualiza codul de server
pentru a identifica posibile probleme ar putea să ajute. Astfel captarea
circuitul ce crează valoare pentru noi ar putea să însemne să avem acces
asupra tuturor elementelor pe care le folosim, de la infrastructură, la servere și baze
de date pentru a nu crea timpi morți așteptând alte persoane să rezolve probleme
ce ne impactează în mod direct.

Îmbunătățirea performanței nu se rezumă doar asupra lucrurilor care
generează profit, dar și menținerea echipelor „în formă” din punct de vedere
mintal. Dezvoltarea de software include probleme ce sunt taxante din punct de vedere
cognitiv, din acest motiv crearea unui mediu relaxant poate ajuta în randamentul personalului.
Însă în același stil putem dorii să sporim creativitatea angajaților. Multe
din diferite idei pornesc de la echipele de inginerie ce se confruntă cu anumite
probleme și vor să creeze soluții pentru a le rezolva, de exemplu la Gmail a pornit de la
un angajat frustrat cu soluția de mail pe care trebuia să o folosească și a dezvoltat
în timpul propriu acea problemă. Unele companii își asignează timp, o zi pe lună,
o săptămână la câteva săptămâni în care angajații lucrează la aceste proeicete
secundare care probabil nu vor genera profit însă care îi ajute să se dezvolte.

Un alt aspect al cultului organizațional ce se poate schimba reprezintă modul
în care aceasta tratează comunitatea, acest lucru referindu-mă la soluțiile open source
pentru diferite produse dar și interacțiunea cu alte companii.
În mediul de afaceri, menținerea avantajului competitiv reprezintă una dintre
prioritățile pe care se axează, pentru a ne diferenția de competitori. Însă
acest lucru nu ar trebui să se axeze și asupra problemelor pe care le întâlnim,
dezvoltatorii ar trebui să fie încurajați să își exprime problemele pentru a
evita alte persoane de a se bloca asupra lor. În același mod folosirea de
pachete open source duce la optimizarea timpilor de dezvoltare dar și la
calitatea produselor.

O altă practică benefică reprezintă motivarea angajaților în mod corespunzător.
Un angajat fericit va avea randament bun. Managerii consideră că partea financiară,
salariul și diferite bonusuri, sunt suficiente pentru a păstra angajații însă
nu este suficient. Un salariu corespunzător nu reprezintă o așteptare, ci o necesitate,
astfel aceștia ar trebui să fie plătiți minim prețul pieții pentru poziția lor.
Dar și ascultați la diferite nivele, fie prin diferite chestionare sau direct,
aceștia pot fi motivați în diferite forme în funcție de dorințele acestora
iar o coordonare bună va folosi dorințele angajaților în atingerea obiectivelor companiei,
însă acesta nu reprezintă neapărat un aspect legat de DevOps ci unul de bună practică în management.

Monitorizarea este un alt aspect important în DevOps, însă nu ar trebui să fie
o surpriză. O companie este evaluată asupra diferiților indicatori de performanță,
iar aceasta se va extinde și la echipele din cadrul acestora. Însă indicatorii
de performanță la nivel de echipă pot fi diferiți, întrucât aceștia au scopul
de a îmbunătății perfomanța echipelor, iar colectarea acestora poate fi făcută în mod diferit
în funcție de procesele firmei. Însă aceste date nu ar trebui să fie folosite
pentru a evalua angajații ci strict pentru a îmbunătății perfomanța echipei.
Atunci când vorbim de performanța unei singure persoane, aceasta nu se traduce
în performanța echipei, iar în general putem să ajungem la crearea de procese
cu scopul perfecționării.

Un aspect important al culturii reprezintă modul de gândire asupra eșecului.
În general, companiile ar încerca să ascundă și să prevină orice problemă cu
produsul pe care îl crează, însă este ineficient să gândim în acest mod, din același
mod în care nu ar trebui să eficientizăm un proces pe care nu știm dacă are nevoie să
fie eficientizat. Astfel o metodă mai bună de combatere a problemelor asociate
produsului este să creăm procese rapide de redresare a acestora prin posibilități
de întoarcere la versiuni anterioare sau de aplicare a unor hotfix-uri cât mai rapid, astfel scăzând
timpul mediu de reparare (MTTR) dar încercand să minimizăm timpul mediu dintre eșecuri (MTBF)
și timpul mediu de eșec (MTTF), de asta e necesar să colectăm fișiere de jurnalizare
și metrici și implemetarea de sisteme ce au ca scop analizarea acestor date
și să încerce să prezică posibilele eșecuri. De asemenea este important să
implementăm sisteme ce minimizează impacul creat de apariția unui erori ce au ca scop
minimizarea posibilelor pierderi din apariția lor, prin diferite
metode precum eșuare rapidă prin care componentele din jur detectează eroare unui sistem
de care depind și își schimbă modul în care se comportă dar și implementarea
unui sistem eficient de avertizare ce poate să ilustreze situația reală a sistemului
și gravitatea acestuia.

Eșecul este inevitabil, iar modul în care îl tratăm este important. Un alt aspect important
în cultura organizatorică este păstrarea unei atitudini în care nu se atribuie vina către
o anumită persoană („blameless”), oamenii fac greșeli și nu le fac în mod intenționat.
Apariția unui incident trebuie să fie urmată de o ședintă de revizie a modului
cum a decurs. Aceasta poate include mai multe detalii precum crearea unui cronologii
a evenimentelor ce s-au desfășurat. Însă aceasta nu va ajuta la descoperirea unei cauze
principale, ce este o practică ce ar trebui să fie evitată. J. Paul Reed mentionează
că găsirea unei astfel de cauze reprezintă doar locul în care te oprești din a investiga mai departe.
Aceste ședințe ar trebui să fie urmate după rezolvarea incidentului și să aibă ca scop
doar îmbunătățirea procesului pe viitor și nu ca motivație negativă. Personalul
ce este amenințat nu va încerca să facă schimbări, astfel nu vor avea oportunitatea de a
crește și de a se perfecționa.
